

\begin{enumerate}[label=\arabic* --- ]\bfseries
	
	\item Introduction de la phase géométrique
	\begin{enumerate}[label=\arabic{enumi}.\arabic* --- ]
		
		\item Un peu d'analyse temps fréquence
		
		\item Les phases totale et dynamique
		
		\item Phase Géométrique
	\end{enumerate}	
	
	\item Décomposition des signaux multivariés
	\begin{enumerate}[label=\arabic{enumi}.\arabic* --- ]
		
		\item Signal AM-FM-PM bivarié
		\begin{itemize} \normalfont
			
			\item la formule avec un peu d'explication sur d'où ca vient
			
			\item les hypothèses sur la décompositions via Bedrosian
			
			\item interprétation géométrique
			
			\item projection sur la sphère avec Stokes et tout
			
		\end{itemize}
		
		\item Calcul de la phase géom dans ce cas
		
		\item Généralisation au plus haute dimension
		\begin{itemize}\normalfont
			
			\item Un mot sur le cas trivarié par Lilly \cite{lilly_modulated_2011}.
			
			\item Lefevre à discuter des généralisations \cite[sec. I.3]{lefevre_polarization_2021} avec des expo et algèbre Clifford : trop de contrainte sur les dimensions des signaux.
			
			\item Généralisation par rotation du plan de polar : boff parce que pas calculable en générale et mais surtout on manque le plus important, ca savoir.... 
			
			\item La phase géo est invariante par transfo de jauge, donc il faut faut faire apparaître $\PC{n-1}$ dans la décomposition.
			
			\item Remarque : c'est le cas en bivarié car $\PC{1}\cong \S{2}$ !
			
			\item S'il faut vraiment se pencher sur $\PC{n-1}$, allons-y (transition partie suivante)
			
		\end{itemize}
	\end{enumerate}
\end{enumerate}

\begin{enumerate}[label=\Alph* --- ] \bfseries
	\item Plus de détail sur l'analyse temps-fréquence
	\begin{enumerate}[label=\Alph{enumi}.\arabic* --- ]
		
		\item Calculation trick
		
		\item Transfo en SA et Bedrisian
		
	\end{enumerate}
\end{enumerate}

\newpage


En traitement du signal, la phase d'un signal est intrinsèquement lié à la notion de fréquence instantanée, qui joue un rôle important en analyse temps-fréquence. 
C'est donc de point que commencera notre discussion pour introduire la phase géométrique.
Pour cela, seront rapidement introduit quelques notions et résultats d'analyse temps-fréquence dans le cas univarié (sec. \ref{sec:ana_temp-freq}). Suite à quoi sera définie la phase instantanée pour le cas multivarié (sec. \ref{sec:freq_instant}), qui permettre enfin de mettre en évidence la phase géométrique (sec. \ref{sec:intro_phaseg}).
\\

Dans une seconde partie, seront introduit les signaux bivarié dit AM-FM-PM, dont la phase géométrique sera calculer explicitement (sec. \ref{sec:AM-FM-PM}). Cela permettre de mettre en évidence certaines de ses propriétés, ce qui permettre ensuite de discuter des généralisations des signaux AM-FM-PM au delà du cas bivarié (sec. \ref{sec:}) et de discuter de la pertinence ces décompositions pour étudier la phase géométrique (sec. \ref{sec:}).

\section{Introduction de la phase géométrique} \label{sec:intro_phaseg}




\subsection{Cas univarié : signaux AM-FM} \label{sec:ana_temp-freq}

%\subsubsection{Quelque notion d'analyse temps-fréquence}

En traitement du signal, l'analyse fréquentielle par la transformée de Fourier est un incontournable. 
Seulement, cette transformation fait perdre toute notion temporelle : si l'étude du spectre du signal permet de dire quelles fréquences apparaissent dans le signal, elle ne permet pas de dire à quel moment. 
C'est en réponse à cela, entre autre, qu'est développé l'analyse temps-fréquence. 
\\
A cette fin, sont définies des paramètres associées aux fréquences mais avec une dépendance en temps :\par

\begin{definition}[Paramètres instantanées] \label{def:param_instant}
	Soit $x$, est un signal complexe écrit sous forme exponentielle :
	\begin{align}
		x\ &:\ \begin{aligned}\R\ &\lr\qquad \C \\
			t\ &\longmapsto\ a(t)e^{i\phi(t)}
		\end{aligned}  &  \text{où }\quad a(t)\in\R^+\quad &\text{et}\quad \phi(t)\in\R
	\end{align}
	$a$ est appelé \emph{amplitude instantanée} du signal, $\nicefrac{1}{2\pi}\phi'$ sa \emph{fréquence instantanée} et sa \emph{phase instantanée} est définie, modulo un choix de phase initiale, par :
	\begin{equation} \label{eq:phasei}
		\phasei(x,t_0,t) = \phi(t) - \phi(t_0)
	\end{equation}
\end{definition}
\skipl 

Pour les signaux réels, ces notions sont moins évidentes à définir puisqu'elles demandent d'écrire les signaux sous la forme :
\[x(t) = a(t) \cos\phi(t)\]
\\
Auquel cas, le choix de la pair $(a,\phi)$ n'est pas unique. Il existe tout de même un ``bon'' choix de telle pair dans le cas des signaux dit AM-FM :
\begin{definition}[Signal AM-FM]
	Un signal réel de la forme :
	\begin{align}
		x\ &:\ \begin{aligned}\R\ &\lr\qquad \R \\
			t\ &\longmapsto\ a(t) \cos\phi(t)
		\end{aligned}  &  \text{où }\quad a(t)\in\R^+\quad
	\end{align}
	est dit \emph{AM-FM} (\emph{amplitude modulated - frequency modulated}) si $a$ et $\cos\phi$ admettent des transformée de Fourier et si de plus la première a un spectre concentré sur les bases fréquences, la seconde concentré sur les hautes fréquences et que les deux ne se chevauche pas.
	Formellement, ces conditions demande qu'il existe $\lambda\in\R^+$ telle que :
	\begin{equation}\label{eq:condi_AM-FM}
		\supp \Fou{a} \subset [-\lambda, \lambda],\quad \supp \Fou{\cos\phi} \subset \R\setminus[-\lambda,\lambda]
	\end{equation}
	Dans ce cas, $a$ et $\phi$ donne lieu au même définition que pour le cas complexe.
\end{definition}
\skipl
Ces conditions sont liées au théorème de Bedrosian, et plus de détail se trouve dans l'annexe \ref{}. Pour le dire rapidement, elles évitent que toutes les fréquences du signal se trouve dans l'amplitude $a$ dans la décomposition $(a,\phi)$, auquel cas, $x$ n'aurait ``pas de fréquence'' au sens où $\phi$ pourrait être choisie constante, voir nulle.
\\
Sous ces conditions, $x$ peut être vu comme le signal complexe $\SA{x}$ telle que :
\[\SA{x}(t) = a(t) e^{i\phi(t)}= a(t)\cos\phi(t) + ia(t)\sin\phi(t) = x  + i\Im m \SA{x}\]
L'on parle alors de transformée en \emph{signal analytique} et $\SA{x}$ a naturellement les mêmes paramètres instantanée que $x$.
\\


%\subsubsection{Le problème des signaux multivariés}

L'intérêt d'introduire toutes ces notions est que les signaux analytiques souffre du même problème que les signaux réels. 
En effet, en écrivant un signal $\x$ sous la forme :
\[\forall t\in\R,\qquad 
\x(t) = \begin{pmatrix} A_1(t)e^{i\Phi_1(t)} \\ A_2(t)e^{i\Phi_2(t)} \\ \vdots \\ A_n(t)e^{i\Phi_n(t)}
\end{pmatrix}\]
\\
Le fait que $\x$ soit à valeur dans $\C^n$ impose un choix naturel d'amplitude instantanée : sa norme. Pour ce qui est de la phase instantanée, en revanche, n'importe qu'elle choix de $\phi$ convient \apriori~:
\[\forall t\in\R,\qquad 
\x(t) = \begin{pmatrix} A_1(t)e^{i\phi_1(t)} \\ A_2(t)e^{i\phi_2(t)} \\ \vdots \\ A_n(t)e^{i\phi_n(t)} \end{pmatrix}
= a(t)e^{i\phi(t)}\begin{pmatrix} a_1(t)e^{i\psi_1(t)} \\ a_2(t)e^{i\psi_2(t)} \\ \vdots \\ a_n(t)e^{i\psi_n(t)} \end{pmatrix}
\qquad\text{ avec }\qquad 
\left\{ \begin{aligned}
	& a(t) = \| \x(t) \|_2 \\
	& \big\|(a_i)_{1\leq i\leq n}\big\|_2 = 1 \\
	& \phi_i = \phi + \psi_i \end{aligned}\right.\]
\\
Il suffit que les $\psi_i$ soient ajustés pour assurer que $\ \phi_i = \phi + \psi_i$.
\\
\begin{remarque}
	À noter, que si $a$ et $\phi$ sont correspondent respectivement à une amplitude et une phase, le vecteur restant $\big( a_ie^{\phi_i} \big)_{1\leq i\leq n}$ correspond à un vecteur de polarisation, sur lequel nous reviendrons dans la \cref{sec:AM-FM-PM} suivante.
\end{remarque}



\subsection{Phase et fréquence instantanée de signal multivarié }\label{sec:param_instant_nvar}

On se propose ici de définir la phase instantanée comme suit :
\begin{definition}[Phase dynalique/instantanée] \label{def:phase_d}
	La \emph{phase instantanée} ou \emph{dynamique} (à l'instant $t$ partant du $t_0$) d'un signal multivarié $\x = a\big(a_ie^{i\phi_i}\big)_{1\leq i\leq n} \in \conti[1]{\R}{\C^n}$ quelconque, est définie par la formule :
	\begin{equation} \label{eq:phase_d}
		\forall t_0, t\in\R, \quad \phased(\x, t_0,t) \defeq \int_{t_0}^t \frac{\Im m \big\langle \dot{\x}(s) , \x(s) \big\rangle}{\|\x(s)\|^2} ds =\sum_{i=1}^n \int_{t_0}^t a_i(s)^2 \phi_i'(s)ds
	\end{equation}
	On s'autorisera à omettre les paramètres de $\phased$ lorsque cela ne prête pas à confusion.
\end{definition}

\begin{remarque}
	Outre l'aspect variationnelle de cette formule, le terme ``dynamique'' viens du fait que, lorsque $\x$ suit une équation de Schrödinger :
	\begin{equation}\label{eq:schrodinger}
		i\frac{d \x(t)}{dt} = h\x(t)
	\end{equation}
	la dérivée $\dot{\x}$ dans la formule \eqref{eq:phase_d} ci-dessus se voit remplacé par l'hamiltonien $h\x$ {\normalfont \cite[sec. 2]{bohm_geometric_2003}, \cite[p.~215]{mukunda_quantum_1993}}, donnant :
	\[\phased' = -i \int_{t_0}^t\frac{\big\langle h\x(s) , \x(s) \big\rangle}{\|\x(s)\|^2} ds\] 
	\\
	Sachant que $\x$ n'a aucune raison de suivre une telle équation dans notre cas, poser $h = i\frac{d}{dt}$ enlève toute contrainte, auquel cas $\phased'=\phased$.
\end{remarque}
\skipl

Cela étant, deux arguments sont donnés pour motiver cette définition :
\\

\subsubsection*{Argument variationnelles}

Le premier, fortement inspirée par les travaux de Lilly \& Olhede  \cite{lilly_analysis_2012}, consiste à généraliser la condition \eqref{eq:condi_AM-FM} de séparation haute/basse fréquences sur les signaux AM-FM.
Pour cela, l'on commence par faire apparaître une phase $\phi$ --- pour l'instant inconnue --- en écrivant $\x$ sous la forme :
\[\forall t\in\R,\qquad \x(t) = e^{i\phi(t)} e^{-i\phi(t)} \x(t) \defeq e^{i\phi(t)} \bf{y}(t)\]
\\
Si $\phi$ est bien choisie, alors $\bf{y}$ ne devrait contenir que les informations associées à l'amplitude et la polarisation de $\x$. Or, conformément à la condition \eqref{eq:condi_AM-FM}, la phase doit contenir les hautes fréquences du signal et, inversement, les basses fréquences doivent se trouver dans ce qui reste. 
\\
La fréquence donnant, pour le dire vite, la vitesse d'ondulation, la contrainte sur $\x$ va être de limite les variations de  $\bf{y}$. Concrètement, $\phi$ doit être choisie de sorte à minimiser la dérivée $\dot{\bf{y}}$ :
\[\forall t\in\R,\qquad \phi(t) = \argmin{\theta(t)}{\big\|\dot{\bf{y}}(t)\big\|_2}^2 = \argmin{\theta(t)}{\Big\|e^{-i\theta(t)}\big(\dot{\x}(t) - i\theta(t)'\x(t)\big)\Big\|_2}^2 = \argmin{\theta(t)}{\big\|\dot{\x}(t) - i\theta'(t)\x(t)\big\|_2}^2\]
\\
La contrainte ne dépendant que de la dérivée $\theta'$, on se ramène à :
\[\min_{\theta(t)}{\|\dot{\bf{y}}(t)\|_2}^2 = \min_{\theta'(t)}{\big\|\dot{\x}(t) - \theta'(t) \x(t)\big\|_2}^2\]
\\
En rappelant que $\frac{d}{dx}{\big\|f(x)\big\|_2}^2 = 2\Re e\big\langle f(x), f'(x)\big\rangle$, il vient que ce minimum\footnote{\itshape
	L'extremum obtenu est l'unique minimum globale puisque $t\longmapsto \|at + b\|^2$ est strictement convexe pour $a\neq0$.}
est atteint par $\phi'(t)$ à condition que :
\begin{align*}
	\frac{d}{d\phi'}{\big\| \dot{\x} - i\phi' \x\big\|_2}^2 = 0 \quad \Llr\quad
	0 &= 2\Re e\left\langle  \dot{\x} - i\phi' \x ,  \frac{d}{d\phi'}\big(\dot{\x} - i\phi' \x\big)\right\rangle \\
	&= 2\Re e\big\langle  \dot{\x} - i\phi' \x ,  - i \x\big\rangle \\
	&= 2\Re e\Big(i\big\langle  \dot{\x} ,  \x\big\rangle\Big) + 2\phi'\Re e\big\langle   \x ,  \x\big\rangle\\
	&= -2\Im m\big\langle  \dot{\x} ,  \x\big\rangle + 2\phi'{\| \x\|_2}^2
\end{align*}
Ainsi $\displaystyle \ \phi' = \frac{\Im m\big\langle  \dot{\x} ,  \x\big\rangle}{{\| \x\|_2}^2}\ $ et :
\begin{equation}\label{eq:phas_inst_v1}
  \phi(t) = \Im m\int_{t_0}^t \frac{\big\langle \dot{\x}(s) , \x(s) \big\rangle}{\|\x(s)\|^2} ds = \phased(\x,t_0,t)
\end{equation}
\\

\subsubsection*{Arguments des moyennes}

Autre argument, cette fois inspiré de \cite{cano_mathematical_2022}, ce base sur la notion de fréquence moyenne.
D'abord dans le cas d'un signal complexe univarié, sont définies les fonctions de densités d'énergie (resp. d'énergie spectale) comme :
\begin{align}\label{eq:densi_dE}
	\densit\ &:\quad \begin{aligned}\R\ &\lr\quad \R^+ \\ t\ &\longmapsto\ \big|x(t)\big|^2 \end{aligned}  
	&
	\text{resp.}\qquad \densis\ &:\quad \begin{aligned}\R\ &\lr\quad \R^+ \\ \nu\ &\longmapsto\ \big|\fou{x}(\nu)\big|^2 \end{aligned}
\end{align}
\\
À partir de ces dernières est définie la fréquence moyenne de $x$ comme comme l'espérance de $\densis$, $\esp[\densis]{\nu}$. Cette fréquence moyenne est lié à la fréquence instantanée par la formule :\footnote{cette formule de généralise à tout les moments de $\densis$ et existe également pour les moments de $\densit$, voir \cite[sec. 1.4]{cohen_time_1995} pour une démonstration ``à la physicienne''}
\begin{equation}\label{eq:esp_freq}
	\esp[\densis]{\nu} = \frac{1}{2\pi}\int_\R \phi'(t)\densit(t)dt = \frac{1}{2\pi} \esp[\densit]{\phi'}
\end{equation}
\\
Dans le cas d'un signal $\x=(x_i)_{1\leq i\leq n}$ multivarié, les densités d'énergies se définissent comme :
\begin{align*}%\label{eq:densi_dEi}
	\densit_i\ &:\quad \begin{aligned}\R\ &\lr\quad \R^+ \\ t\ &\longmapsto\ \big|x_i(t)\big|^2 = a(t)^2 a_i(t)^2 \end{aligned}  
	&
	\densis_i\ &:\quad \begin{aligned}\R\ &\lr\quad \R^+ \\ \nu\ &\longmapsto\ \big|\fou{x}_i(\nu)\big|^2 \end{aligned} \\ \\
	%\label{eq:densi_dE-mv}
	\densit\ &:\quad \begin{aligned}\R\ &\lr\quad \R^+ \\ t\ &\longmapsto\ \big\|\x(t)\big\|^2 = \sum_{i=1}^n \densit_i(t) \end{aligned}  
	&
	\densis\ &:\quad \begin{aligned}\R\ &\lr\quad \R^+ \\ \nu\ &\longmapsto\ \big\|\fou{\x}(\nu)\big\|^2 = \sum_{i=1}^n \densis_i(t) \end{aligned}	
\end{align*}
Le second argument consiste alors à dire que l'égalité des moments $\eqref{eq:esp_freq}$ doit resté vrai dans le cas multivarié. Cela assure au moins que la fréquence instantanée de $\x$, $\nicefrac{1}{2\pi}\phi'$, à pour moyenne la fréquence moyenne en sens de Fourier.
\\

En appliquant la formule \eqref{eq:esp_freq} au $\densis_i$, et en notant toujours $\x = a\big(a_ie^{i\phi_i}\big)_{1\leq i\leq n}$, on obtient :
\begin{align*}
	\esp[\densis]{\nu} = \int_\R \nu\densis(\nu)d\nu &= \int_\R \nu\sum_{i=1}^n \densis_i(\nu) d\nu \\
	&= \sum_{i=1}^n\esp[\densis_i]{\nu} \\
	&= \sum_{i=1}^n\frac{1}{2\pi}\int_\R \phi_i'(t)\densit_i(t)dt \\
	&= \frac{1}{2\pi}\int_\R a(t)^2\sum_{i=1}^n\phi_i'(t)a_i(t)^2 dt 
	\\ &= \frac{1}{2\pi} \esp[\densit]{\sum_{i=1}^n \phi_i'{a_i}^2}
\end{align*}
\\
Ce qui mène à poser $\displaystyle \ \sum_{i=1}^n \phi_i'(t){a_i}^2(t)\ $ pour la fréquence instantanée, avec la phase associée :
\begin{equation}\label{eq:phas_inst_v1}
	\phi = \int_{t_0}^t \sum_{i=1}^n \phi_i'(s){a_i}(s)^2ds 
	= \sum_{i=1}^n \int_{t_0}^t \phi_i'(s){a_i}(s)^2ds 
	%= \sum_{i=1}^n \esp[\nicefrac{\densit_i}{\densit}]{\phi_i'}
\end{equation}
\\

Formule qui concorde bien avec celle de la phase dynamique une fois explicité :
\begin{align*}
	\Im m\frac{\big\langle \dot{\x}(t) , \x(t) \big\rangle}{\|\x(t)\|^2} &= \Im m\left( \frac{1}{a(t)^2} \sum_{i=1}^n \Big( \big(aa_i\big)'(t) +a(t)a_i(t)i\phi_i'(t)\Big)e^{i\phi_i(t)}\congu{a(t)a_i(t)e^{i\phi_i(t)}} \right) \\
	&=\frac{1}{a(t)^2}  \Im m\left( \sum_{i=1}^n a(t)a_i(t)\big(aa_i\big)'(t) +ia(t)^2a_i(t)^2\phi_i'(t) \right) \\
	&= \frac{1}{a(t)^2} \sum_{i=1}^n a(t)^2a_i(t)^2 \phi_i'(t) \\
	&= \sum_{i=1}^n a_i(t)^2 \phi_i'(t)
\end{align*}
D'où
\[\Im m\int \frac{\big\langle \dot{\x}(s) , \x(s) \big\rangle}{\|\x(s)\|^2} ds = \int \sum_{i=1}^n a_i(s)^2 \phi_i'(s) = \sum_{i=1}^n \int a_i(s)^2 \phi_i'(s)ds\]
\\



\subsection{Apparition de la phase géométrique --- A REFAIRE}\label{subsub:intro_phaseg}

Pour rentre compte de la pertinence de cette expression, commençons par noter qu'il existe une autre façon standard de définir la phase d'un signal, la \emph{phase totale} :
\begin{equation}\label{eq:phase_t}
	\phaset(\x, t_0, t) \defeq \arg\big\langle \x(t), \x(t_0)\big\rangle
\end{equation}
Il n'est pas clair, dans un cadre générale, comment et pourquoi cela s'interprète bien comme une phase et c'est encore pire lorsque l'on explicite sa valeur :
\begin{align*}
	\phaset(\x,t_0, t) &= \arg \left( \sum_{i=1}^n a_i(t)a_i(t_0)e^{i(\phi_i(t)-\phi_i(t_0))} \right) \\
	&= \phased + \arg \left( \sum_{i=1}^n a_i(t)a_i(t_0)e^{i(\alpha_i(t)-\alpha_i(t_0))} \right)  &  &\text{où } \phi_i = \phased + \alpha_i  \\
	&= \phased + \arctan \left( \frac{\sum_i a_i(t)a_i(t_0)\sin\big( \alpha_i(t)-\alpha_i(t_0)\big)}{\sum_i a_i(t)a_i(t_0)\cos\big( \alpha_i(t)-\alpha_i(t_0)\big)}  \right)
\end{align*}
\\

Cela étant dit, si $\x$ est cyclique à une phase près, cette formule fait plus sens. C'est-à-dire lorsque, entre deux instant $t_0$ et $t$ donnés, $\x$ vérifie :
\[\exists \theta\in\R\ |\quad \x(t) = e^{i\theta} \x(t_0)\]
Dès lors, la phase totale donne bien :
\[\arg\big\langle \x(t), \x(t_0)\big\rangle = \arg\big\langle e^{i\theta} \x(t_0), \x(t_0)\big\rangle = \theta\]
\\
Dans le cas univarié, la phase instantanée vaut également $\theta$, ce qui n'est plus le cas dès que $n\geq 2$ (voir \cref{fig:calc_diff_phases}, ci-dessous). Apparaît alors une nouvelle phase qui est dû au caractère multivarié du signal : la phase géométrique introduite au début du mémoire.
\\
\begin{figure}[h]
	\includegraphics[width=0.6\textwidth]{fig/placeholder}
	\caption{Sur le graphe de gauche, le signal $\x$ à valeur dans $\R^2$ et dans celui de droite la calcul de la phase dynamique, totale et de leur différence. Résultat tiré des simulation de Le Bihan \etal \cite{le_bihan_modephysiques_2023}}
	\label{fig:calc_diff_phases}
\end{figure}

En analyse temps-fréquence, la phase instantanée d'un signal complexe $x: \R\lr\C$ par son argument, modulo un choix de phase initial. En clair, si $x$ s'écrit $\ x(t) = a(t) e^{i\phi(t)}$, alors :
\begin{equation} \label{eq:phasei}
	\phasei(x,t_0,t) = \phi(t) - \phi(t_0)
\end{equation}
\\
Phase qui peut encore s'écrire :
\[\phasei(x,t_0,t) = \arg x(t)\congu{x(t_0)}\]
\\
La phase totale peut-être vu comme une généralisation de cette formule au signaux multivarié, \ie~à valeur dans $\C^n$.


\begin{equation}\label{eq:AM-FM-PM_2var}
	\forall t\in\R,\quad \x(t) = a(t)e^{i\varphi(t)} R_{\theta(t)} \begin{pmatrix} \cos\chi(t) \\ -i\sin\chi(t) \end{pmatrix}
\end{equation}




\section{Décomposition des signaux multivariées} \label{sec:AM-FM-PM}
