
\begin{enumerate}[label=\arabic* --- ]\bfseries
	
	\item Introduction de la phase géométrique
	\begin{itemize}\normalfont
		
		\item En analyse temps-fréquence, il est utile de définir une notion de fréquence instantanée (même si ca peut pas avoir de sens) \cite{cohen_time_1995} et elle est donnée par la dérivée à $2\pi$ près de l'argument. La phase vient alors comme primitive de la fréquence instantanée. Autrement dit, c'est l'argument à un choix de phase initiale près.
		
		\item Comme nous on étudie l'évolution d'une phase au cours du temps, c'est l'analyse temps-fréquence est précisément le cadre qu'il nous faut et c'est au regard de cette dernière que nous allons interpréter les phase totale, dynamique et géométrique.
		
	\end{itemize}
	\begin{enumerate}[label=\arabic{enumi}.\arabic* --- ]
		
		\item Fréquence instantanée
		\begin{itemize}
			
		\end{itemize}
		
		\item Phase totale
		\begin{itemize} \normalfont 
			\item petit blabla sur le produit hermitien (convention, interprétation par rapport au produit scalaire)
			
			\item interprétation de la phase totale comme angle entre vecteur (dans $\C^1$ puis généralisé)
			
			\item \textit{jsp}
		\end{itemize}
		
		\item Phases dynamique (via fréquence instantanée)
		\begin{itemize} \normalfont
			
			\item Généralisation de la condition de Bedrosian 
			
			\item en plus ca colle la phase instant dans le cas univariée
			
			\item (annexes : les arguments pour l'appellation ``phase instantanée'')
			
		\end{itemize}
		
		\item Phase Géométrique
		\begin{itemize} \normalfont
			
			\item cas univarié : égalité des deux fréquences
			
			\item arrivée de la phase géométrique
			
			\item invariance par transformation de jauge 
			
			\item son calcul pour les AM-FM-PM
			
		\end{itemize}
	\end{enumerate}	
	
	\item Cas particulier de signal
	\begin{enumerate}[label=\arabic{enumi}.\arabic* --- ]
		
		\item Signal AM-FM-PM bivarié
		\begin{itemize} \normalfont
			
			\item la formule avec un peu d'explication sur d'où ca vient
			
			\item les hypothèses sur la décompositions via Bedrosian
			
			\item interprétation géométrique
			
			\item projection sur la sphère avec Stokes et tout
			
		\end{itemize}
		
		\item Calcul de la phase géom dans ce cas
		
		\item Généralisation au plus haute dimension
		\begin{itemize}\normalfont
			
			\item Un mot sur le cas trivarié par Lilly \cite{lilly_modulated_2011}.
			
			\item Lefevre à discuter des généralisations \cite[sec. I.3]{lefevre_polarization_2021} avec des expo et algèbre Clifford : trop de contrainte sur les dimensions des signaux.
			
			\item Généralisation par rotation du plan de polar : boff parce que pas calculable en générale et mais surtout on manque le plus important, ca savoir.... 
			
			\item La phase géo est invariante par transfo de jauge, donc il faut faut faire apparaître $\PC{n-1}$ dans la décomposition.
			
			\item Remarque : c'est le cas en bivarié car $\PC{1}\cong \S{2}$ !
			
			\item S'il faut vraiment se pencher sur $\PC{n-1}$, allons-y (transition partie suivante)
			
		\end{itemize}
	\end{enumerate}
\end{enumerate}

\newpage





\section{Introduction de la phase géométrique} \label{sec:AM-FM-PM}

La formule phase géométrique telle que donné en introduction possède deux composantes appelées phase totale et phase dynamique :
\begin{equation}\label{eq:phase_t}
	\phaset = \arg\big\langle \bf{x}(t), \bf{x}(t_0)\big\rangle
\end{equation}
\begin{equation}\label{eq:phase_d}
	\phased = \Im m\int_{t_0}^t \frac{\big\langle \dot{\bf{x}}(s) , \bf{x}(s) \big\rangle}{\|\bf{x}(s)\|^2} ds
\end{equation}
De sorte que :
\begin{equation}\label{eq:phase_g}
	\phaseg = \phaset - \phased = \arg\big\langle \bf{x}(t), \bf{x}(t_0)\big\rangle - \Im m\int_{t_0}^t \frac{\big\langle \dot{\bf{x}}(s) , \bf{x}(s) \big\rangle}{\|\bf{x}(s)\|^2} ds
\end{equation}
\\
Pour comprendre d'où vient cette formule, dans cette partie seront d'abord expliquées les phases totale (sec. \ref{sec:phase_t}) et dynamique (sec. \ref{sec:phase_d}) avant de revenir sur la phase géométrique (sec. \ref{sec:phase_g}).



\subsection{Phase totale d'un signal} \label{sec:phase_t}

En analyse temps-fréquence, la phase instantanée d'un signal complexe $x: \R\lr\C$ par son argument, modulo un choix de phase initial. En clair $x$ s'écrit $x(t) = a(t) e^{i\phi(t)}$, alors :
\begin{equation} \label{eq:phasei}
	\phasei(x,t_0,t) = \phi(t) - \phi(t_0)
\end{equation}
\\
Phase qui peut encore s'écrire :
\[\phasei(x,t_0,t) = \arg x(t)\congu{x(t_0)}\]
\\
La phase totale peut-être vu comme une généralisation de cette formule au signaux multivarié, \ie~à valeur dans $\C^n$.







\newpage


\begin{equation}\label{eq:AM-FM-PM_2var}
	\forall t\in\R,\quad \bf{x}(t) = a(t)e^{i\varphi(t)} R_{\theta(t)} \begin{pmatrix} \cos\chi(t) \\ -i\sin\chi(t) \end{pmatrix}
\end{equation}
