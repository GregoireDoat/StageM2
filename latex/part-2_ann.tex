
\subsection{Démonstration des résultats \cref{subsec:phase_g2aire}} \label{ann:stokes}

\subsubsection{Formule pour $\bf{\phaseg}$ sur $\bf{\PC{n}}$} \label{ann:proj2phaseg}

Ici $\gamma$ est supposé cyclique et au dessus d'une unique carte $U_i$ (par commodité), avec :
\[\gamma = h_i(\rho, e^{i\theta}) = \sigma_i(\rho) e^{i\theta}\]
\\
Avec ces notations, la phase totale de $\gamma$ va s'écrire :
\[\phaset(\gamma, t_0, t) = t\big( \gamma(t), \gamma(t_0) \big) = \theta(t) - \theta(t_0)\]
\\

Pour ce qui est de la phase dynamique, on comment par calculer la connexion le long de $\gamma$, ce qui nécessite d'écrire :
\begin{align*}
	\dot{\gamma} = \frac{d}{dt} \Big( \sigma_i(\rho) e^{i\theta} \Big) &=  \sigma_{i*}(\dot{\rho}) e^{i\theta} + i\theta' \sigma_i(\rho) e^{i\theta} \\
	&= \sigma_{i*}(\dot{\rho}) e^{i\theta} + (i\theta')^\#\big(\sigma_i(\rho) e^{i\theta}\big)  & \text{par définition de $\#$, \cref{eq:alg2vg}}
\end{align*}
\\
Avec, et sachant les propriétés de $\conn$ (\cref{eq:1-form2conn,eq:1-form_G-inv}, \Cref{def:1-form2conn}), on a :
\begin{align*}
	i\conn_\gamma(\dot{\gamma}) &= i\conn_{\sigma_i(\rho) e^{i\theta}} \Big( \sigma_{i*}(\dot{\rho}) e^{i\theta} + (i\theta')^\#\big(\sigma_i(\rho) e^{i\theta}\big) \Big) \\
	&= \conn_{\sigma_i(\rho) e^{i\theta}} \Big( \sigma_{i*}(\dot{\rho}) e^{i\theta} \Big) + \conn_{\sigma_i(\rho) e^{i\theta}} \Big((i\theta')^\#\big(\sigma_i(\rho) e^{i\theta}\big) \Big) \\
	&= e^{-i\theta} \conn_{\sigma_i(\rho)} \big( \sigma_{i*}(\dot{\rho}) \big) e^{i\theta} + i\theta'
\end{align*}
\\
D'où la phase dynamique (\cref{eq:phased_vgeodiff}) :
\begin{align*}
	\phased(\gamma) = \frac{1}{i}\int_{t_0}^t \conn_{\gamma(s)} \big( \dot{\gamma}(s) \big)ds 
	&= \frac{1}{i}\int_{t_0}^t \Big( \locconn_{i\, \rho(s)}\big(\dot{\rho}(s) \big) + i\theta'(s)\Big)ds \\
	&= -i \oint \locconn_{i\, \rho} (\dot{\rho}) + \theta(t) - \theta(t_0)
\end{align*}
\\
et par conséquent la phase géométrique :
\begin{align*}
	\phaseg(\gamma) &= \phaset(\gamma) - \phased(\gamma) \\
	&= \theta(t) - \theta(t_0) - \left( -i \oint \locconn_{i\, \rho} (\dot{\rho}) + \theta(t) - \theta(t_0) \right) \\
	&= i \oint \locconn_{i\, \rho} (\dot{\rho})
\end{align*}
\\

Maintenant, pour pouvoir appliquer le théorème de Stokes, il faut s'assurer que la variété étudiée est orientable, ce qui est le cas de toute les variétés complexes \cite[sec. 8.4.2]{nakahara_geometry_2003} (y compris $\PC{n}$). Ainsi, pour peu que $\rho$ soit suffisamment régulière, le théorème s'applique et :
\[\oint \locconn_{i}  = \iint_\Sigma d\locconn_i\]
\\
avec $\Sigma$ une surface de$\PC{n}$ de bord $\rho$.
\\



\subsubsection{Dérivation de $\bf{\phaseg}$ en tant qu'aire de $\bf{\PC{n}}$} \label{ann:phaseg=aire}

Par définition, sur l'ouvert $U_i$, la 1-forme de connexion local est définie par :
\[\locconn_i = {\sigma_i}^*\conn = \conn\circ \sigma_{i*}\]
\\
Soit, $\forall w\in U_i,\ \forall \bf{v}\in\tg[w]{\PC{n}}$ :
\[\locconn_i(w)\bf{v}= i\Im m \big\langle \sigma_{i*}(\bf{v}), \sigma_i(w) \big\rangle\]
\\
où les $\sigma_{i*}$ s'écrivent, $\forall \mu$ :
\begin{align*}
	\mu &\neq i : &  \sigma_i(w)^\mu = \frac{w^\mu}{\sqrt{1 + w^\alpha\congu{w}_\alpha}}\quad \Lr\quad
	\sigma_{i*}(w)^\mu &= \frac{dw^\mu}{\sqrt{1 + w^\alpha\congu{w}_\alpha}} - \frac{w^\mu}{2(1 + w^\alpha\congu{w}_\alpha)^{\nicefrac{3}{2}}}2\Re e(w^\alpha d\congu{w}_\alpha) \\
	&  &  &= \frac{1}{\sqrt{1 + w^\alpha\congu{w}_\alpha}}\left(dw^\mu - w^\mu \frac{\Re e(w^\alpha d\congu{w}_\alpha)}{1 + w^\alpha\congu{w}_\alpha}\ \right) \\
	&  &  &= \frac{1}{\sqrt{1 + w^\alpha\congu{w}_\alpha}}\left(dw^\mu - w^\mu \frac{\congu{w}^\alpha dw_\alpha + w^\alpha d\congu{w}_\alpha}{1 + w^\alpha\congu{w}_\alpha}\ \right)
	\\ \\
	\mu &= i :  &  \sigma_i(w)^\mu = \frac{1}{\sqrt{1 + w^\alpha\congu{w}_\alpha}}\quad \Lr\quad
	\sigma_{i*}(w)^\mu &= - \frac{\Re e(w^\alpha d\congu{w}_\alpha)}{(1 + w^\alpha\congu{w}_\alpha)^{\nicefrac{3}{2}}} = - \frac{\congu{w}^\alpha dw_\alpha + w^\alpha d\congu{w}_\alpha}{(1 + w^\alpha\congu{w}_\alpha)^{\nicefrac{3}{2}}}
\end{align*}
\\
Ce qui donne\footnote{\itshape
	Dans le formule ci-dessous, les 0 et 1 sont placés à la $i^{eme}$ coordonnées.
} :
\begin{align*}
	\locconn_i(w) &= i\Im m \big\langle \sigma_{i*}(w), \sigma_i(w) \big\rangle \\
	&= i\Im m \left\langle \frac{1}{\sqrt{1 + w^\alpha\congu{w}_\alpha}}\left((dw^0,\cdots , 0, \cdots, dw^n) - (w^0,\cdots , 1, \cdots, w^n) \frac{\Re e(w^\alpha d\congu{w}_\alpha)}{1 + w^\alpha\congu{w}_\alpha}\ \right), \frac{(w^0,\cdots , 1, \cdots, w^n)}{\sqrt{1 + w^\alpha\congu{w}_\alpha}} \right\rangle \\
	&= \frac{1}{1 + w^\alpha\congu{w}_\alpha} i\Im m \left( \Big\langle (dw^0,\cdots , 0, \cdots, dw^n), (w^0,\cdots , 1, \cdots, w^n) \Big\rangle \bigg. \right. \\
	&\qquad\qquad\qquad\qquad\qquad - \left.  \frac{\Re e(w^\alpha d\congu{w}_\alpha)}{1 + w^\alpha\congu{w}_\alpha}\Big\langle (w^0,\cdots , 1, \cdots, w^n) , (w^0,\cdots , 1, \cdots, w^n) \Big\rangle \right) \\
	&= \frac{1}{1 + w^\alpha\congu{w}_\alpha} i\Im m \left( dw^\mu\congu{w}_\mu -  \frac{\Re e(w^\alpha d\congu{w}_\alpha)}{1 + w^\alpha\congu{w}_\alpha}\big( w^\nu\congu{w}_\nu +1 \big) \right)
\end{align*}
\\
Enfin, sachant que le second membre dans la partie imaginaire est réel, il vient :
\begin{align*}
	\locconn_i(w) = \frac{1}{1 + w^\alpha\congu{w}_\alpha} i\Im m \left( dw^\mu\congu{w}_\mu -  \frac{\Re e(w^\alpha d\congu{w}_\alpha)}{1 + w^\alpha\congu{w}_\alpha}\big( w^\nu\congu{w}_\nu +1 \big) \right) 
	&= \frac{1}{1 + w^\alpha\congu{w}_\alpha} i\Im m \big( dw^\mu\congu{w}_\mu \big) \\
	&= \frac{1}{1 + w^\alpha\congu{w}_\alpha} \frac{dw^\mu\congu{w}_\mu -  d\congu{w}^\nu w_\nu }{2}
\end{align*}
\\

Maintenant, pour avoir les coefficients de $d\locconn_i$, il faut calculer respectivement :
\begin{align*}
	\partial_\lambda \locconn_{i\, \mu} &= \partial_\lambda \frac{\congu{w}_\mu}{2(1 + w^\alpha\congu{w}_\alpha)}  &  
		\partial_{\congu{\lambda}} \locconn_{i\, \mu} &= \partial_{\congu{\lambda}} \frac{\congu{w}_\mu}{2(1 + w^\alpha\congu{w}_\alpha)}
	\\
	&= \frac{\congu{w}_\mu\congu{w}_\lambda}{2(1 + w^\alpha\congu{w}_\alpha)^2}  &  
		&=  \frac{1}{2(1 + w^\alpha\congu{w}_\alpha)} \Big( \delta_{\lambda\mu} - \frac{\congu{w}_\mu w_\lambda}{1 + w^\alpha\congu{w}_\alpha}\Big) \\
	& &  &= \frac{1}{4} g_{\mu \congu{\lambda}}
	\\ \\
	\partial_\lambda \locconn_{i\, \congu{\nu}} &= \partial_{\lambda} \frac{ -w_\nu}{2(1 + w^\alpha\congu{w}_\alpha)}  &  
		\partial_{\congu{\lambda}} \locconn_{i\, \congu{\nu}} &= \partial_{\congu{\lambda}} \frac{ -w_\nu}{2(1 + w^\alpha\congu{w}_\alpha)}
	\\
	&=  \frac{-1}{2(1 + w^\alpha w_\alpha)} \Big( \delta_{\lambda\nu} - \frac{ w_\nu \congu{w}_\lambda}{1 + w^\alpha\congu{w}_\alpha}\Big)  &
		&= -\frac{ w_\nu w_\lambda}{2(1 + w^\alpha\congu{w}_\alpha)^2} \\
	&= -\frac{1}{4} g_{\lambda \congu{\nu}}
\end{align*}
\\
On remarque alors les coefficient $d\locconn_{i\, \lambda \mu}$ et $d\locconn_{i\, \congu{\lambda \mu}}$ sont symétriques, ce qui fait qu'avec le produit extérieur il s'annule (anti-symétrie). Par exemple :
\[(d\locconn_i)_{\lambda \mu}\, dw^\lambda \wedge dw^\mu = \frac{\congu{w}_\mu\congu{w}_\lambda}{2(1 + w^\alpha\congu{w}_\alpha)^2} dw^\lambda \otimes dw^\mu - \frac{\congu{w}_\mu\congu{w}_\lambda}{2(1 + w^\alpha\congu{w}_\alpha)^2} dw^\mu \otimes dw^\lambda = 0\]
\\
Ce qui mène finalement à :
\begin{align*}
	d\locconn_i &= \frac{1}{4} g_{\mu \congu{\lambda}} d\congu{w}^\lambda \wedge dw^\mu  - \frac{1}{2} g_{\lambda \congu{\nu}} dw^\lambda \wedge d\congu{w}^\nu \\
	&= -\frac{1}{4} \big( g_{\mu \congu{\nu}} dw^\mu \wedge d\congu{w}^\nu  + g_{\mu \congu{\nu}} dw^\mu \wedge d\congu{w}^\nu \big)  &  &\text{par anti-symétrie du produit extérieur}\\
	&= - \frac{1}{2} g_{\mu \congu{\nu}} dw^\mu \wedge d\congu{w}^\nu \\
	&= \i \frac{1}{2} \Omega_{\mu \congu{\nu}} dw^\mu \wedge d\congu{w}^\nu
\end{align*}
\skipl







\subsection{Géodésique de $\bf{\PC{n}}$} \label{ann:geode2PC^n}


\subsubsection{Métrique relevée dans les espaces horizontaux}

D'abord les vecteurs tangent de $\S{n}$ sont séparés en composantes verticale et horizontale :
\begin{align}
	\forall \bf{v}\in \tg[p]{\S{n}},\quad \bf{v} = \bf{v}_H + \omega_p(\bf{v})^\# &= \bf{v}_H + \frac{d}{dt}p\cdot \exp\big(it\Im m \langle \bf{v}, p \rangle\big)\Big|_{t=0} \\
	&= \bf{v}_H  + i\Im m \langle \bf{v}, p \rangle p
\end{align}
\\
Ainsi, $\forall \bf{u},\bf{v}\in \tg[p]{\S{n}}$ :
\begin{align*}
	g_{\pi(p)}( \pi_*\bf{u}, \pi_*\bf{v}) = \big\langle \bf{u}_H, \bf{v}_H \big\rangle &= \big\langle\bf{u} - \conn_p(\bf{u})^\#, \bf{v} - \conn_p(\bf{v})^\# \big\rangle \\
	&= \big\langle\bf{u}, \bf{v} \big\rangle  - \big\langle\bf{u}, \conn_p(\bf{v})^\# \big\rangle - \big\langle \conn_p(\bf{u})^\#, \bf{v} \big\rangle + \big\langle \conn_p(\bf{u})^\#, \conn_p(\bf{v})^\# \big\rangle \\
	&= \big\langle\bf{u}, \bf{v} \big\rangle  - \big\langle\bf{u},  i\Im m \langle \bf{v}, p \rangle p \big\rangle - \big\langle  i\Im m \langle \bf{u}, p \rangle p, \bf{v} \big\rangle + \big\langle  i\Im m \langle \bf{u}, p \rangle p,  i\Im m \langle \bf{v}, p \rangle p \big\rangle \\
	&= \big\langle\bf{u}, \bf{v} \big\rangle  + i\Im m \langle \bf{v}, p \rangle\big\langle\bf{u}, p \big\rangle - i\Im m \langle \bf{u}, p \rangle\big\langle p, \bf{v} \big\rangle - i\Im m \langle \bf{u}, p \rangle i\Im m \langle \bf{v}, p \rangle \big\langle p, p \big\rangle
\end{align*}
\\
Sachant que $\ \|p\|=1\ $ et $\ \Re e \langle \bf{v},p\rangle = 0$, il vient :
\begin{align*}
	g_{\pi(p)}( \pi_*\bf{u}, \pi_*\bf{v}) 
	&= \big\langle\bf{u}, \bf{v} \big\rangle  + i\Im m \langle \bf{v}, p \rangle\big\langle\bf{u}, p \big\rangle - i\Im m \langle \bf{u}, p \rangle\big\langle p, \bf{v} \big\rangle - i\Im m \langle \bf{u}, p \rangle i\Im m \langle \bf{v}, p \rangle \big\langle p, p \big\rangle \\
	&= \big\langle\bf{u}, \bf{v} \big\rangle  -\Im m \langle \bf{v}, p \rangle \Im m\big\langle\bf{u}, p \big\rangle + \Im m \langle \bf{u}, p \rangle \Im m \big\langle p, \bf{v} \big\rangle - \langle \bf{u}, p \rangle \langle \bf{v}, p \rangle\\
	&= \big\langle\bf{u}, \bf{v} \big\rangle -  \langle \bf{u}, p \rangle \langle \bf{v}, p \rangle
\end{align*}
\\

Ce qui donne en coordonnées locales sur $\S{n}$ : 
\[g = \delta_{\mu \nu} dz^\mu d\congu{z}^\nu - \delta_{\mu \beta}z^\mu d\congu{z}^\beta \delta_{\alpha \nu} dz^\alpha \congu{z}^\nu = \big( \delta_{\mu \nu} - z_\nu \congu{z}_\mu \big) dz^\mu d\congu{z}^\nu\]
\skipl





\subsubsection{Ecriture des géodésiques}

\textit{Les calculs de cette section reprenne en partie les calculs de Mukunda \& Simon \cite[sec. 4, p. 219]{mukunda_quantum_1993}.}
\\

Etant donnée, sur une variété $\manu$, une métrique $g$ de symbole de Christoffel associé $\Gamma$, une géodésique $\gamma$ de $\manu$ vérifie \cite{do_carmo_riemannian_1992} :
\begin{equation}\label{eq:EDP2geode}
	\forall \sigma,\quad \ddot{\gamma}^\sigma + \Gamma^\sigma _{\mu \nu} \dot{\gamma}^\mu \dot{\gamma}^\nu = 0
\end{equation}
\\
Pour une variété complexe, les contraintes apportés par les composantes holomorphe  et anti-holomorphe  sont les mêmes. Le système reste donc le même à la différence près que cette fois les symboles de Christoffel vont s'écrire\footnote{\itshape
	Les symétries imposées à $g$ par la forme symplectique $J$ annule la majorité des composantes de $g$ et \afortiori, de $\Gamma$. Voir \cite[sec. 8.4.3]{nakahara_geometry_2003}
} :
\begin{align}\label{eq:symb2Christo}
	\Gamma^\sigma _{\mu \alpha} &= g^{\sigma \congu{\beta}} \partial_{\mu} (g_{\alpha \congu{\beta}})  &  \Gamma^{\congu{\sigma}} _{\congu{\nu \beta}} &= g^{\alpha \congu{\sigma}} \partial_{\congu{\nu}} (g_{\alpha \congu{\beta}}) 
\end{align}
\\
Le système d'EDP \eqref{eq:EDP2geode} s'écrit alors :
\begin{align*}
	\ddot{\gamma}^\sigma + \Gamma^\sigma _{\mu \alpha}\,  \dot{\gamma}^\mu\, \dot{\gamma}^\alpha = 0 
	\quad &\Llr\quad
	\ddot{\gamma}^\sigma + g^{\sigma \congu{\beta}} \partial_{\mu} (g_{\alpha \congu{\beta}}) \, \dot{\gamma}^\mu\, \dot{\gamma}^\alpha = 0 \\
	&\Llr\quad 
	g_{\sigma \congu{\beta}}\, \ddot{\gamma}^\sigma + g_{\sigma \congu{\beta}}\, g^{\sigma \congu{\beta}} \partial_{\mu} (g_{\alpha \congu{\beta}}) \, \dot{\gamma}^\mu\, \dot{\gamma}^\alpha = 0 \\
	&\Llr\quad 
	g_{\sigma \congu{\beta}}\, \ddot{\gamma}^\sigma + \partial_{\mu} (g_{\alpha \congu{\beta}}) \, \dot{\gamma}^\mu\, \dot{\gamma}^\alpha = 0
\end{align*}
\\
Dans le cas de $\VFP$, les $\partial g_{\alpha \congu{\beta}}$ s'écrivent :
\begin{align*}
	\partial_{\mu} (g_{\alpha \congu{\beta}}) &= \partial_{\mu} \big( \delta_{\alpha \beta} - \congu{z}_\alpha z_\beta \big) = - \delta_{\mu\beta} \congu{z}_\alpha   &  
	\partial_{\congu{\nu}} (g_{\alpha \congu{\beta}}) &= \partial_{\congu{\nu}} \big( \delta_{\alpha \beta} - \congu{z}_\alpha z_\beta \big) = - \delta_{\nu \alpha} z_\beta
\end{align*}
\\
Donnant les équations :
\begin{align*}
	\begin{aligned}
		\forall \beta,\quad 0 &= g_{\sigma \congu{\beta}}\, \ddot{\gamma}^\sigma + \partial_{\mu} (g_{\alpha \congu{\beta}}) \, \dot{\gamma}^\mu\, \dot{\gamma}^\alpha \\
		&= \big( \delta_{\sigma \beta} - \congu{\gamma}_\sigma \gamma_\beta \big)\, \ddot{\gamma}^\sigma - \delta_{\mu\beta} \congu{\gamma}_\alpha \, \dot{\gamma}^\mu\, \dot{\gamma}^\alpha \\
		&= \ddot{\gamma}_\beta - \gamma_\beta \langle \ddot{\gamma},\gamma\rangle - \dot{\gamma}_\beta \langle  \dot{\gamma}, \gamma \rangle 
	\end{aligned} \qquad \Llr\quad 
	0 = \ddot{\gamma} -  \langle \ddot{\gamma},\gamma\rangle \gamma -  \langle  \dot{\gamma}, \gamma \rangle \dot{\gamma}
\end{align*}
\\
Où l'équivalence est justifiée par le fait que les composantes anti-holomorphes des $\gamma, \dot{\gamma}, \ddot{\gamma}$ suivent les mêmes contraintes (à conjugaison près) celles holomorphes.

Pour résoudre ce système, le produit hermitien de ce dernier avec $\gamma$ est calculé :
\begin{align*}
	\ddot{\gamma} =  \langle \ddot{\gamma},\gamma\rangle \gamma + \langle  \dot{\gamma}, \gamma \rangle \dot{\gamma} \quad 
	&\Lr\quad \langle \ddot{\gamma}, \gamma \rangle =  \langle \ddot{\gamma},\gamma\rangle \langle \gamma, \gamma \rangle + \langle  \dot{\gamma}, \gamma \rangle^2 \\
	&\Lr\quad 0 =  \langle  \dot{\gamma}, \gamma \rangle
\end{align*}
On retrouve alors le fait que $\dot{\gamma}$ est horizontale et $\ \ddot{\gamma} = \gamma \langle \ddot{\gamma},\gamma\rangle$.
\\
En appliquant à nouveau le produit hermitien mais de l'autre côté, cette fois :
\[\ddot{\gamma} = \gamma \langle \ddot{\gamma},\gamma\rangle\quad 
\Lr\quad \langle \gamma, \ddot{\gamma} \rangle = \langle  \gamma, \gamma \rangle \langle \ddot{\gamma},\gamma\rangle  =  \langle  \ddot{\gamma}, \gamma \rangle\]
\\
Sachant que $\gamma\in\S{n}$, on a alors :
\begin{align*}
	\| \gamma \| = 1\ &\Lr\ \langle \gamma, \dot{\gamma} \rangle + \langle \dot{\gamma}, \gamma \rangle =0 \\
	&\Lr\ \langle \gamma, \ddot{\gamma} \rangle + 2\langle \dot{\gamma}, \dot{\gamma} \rangle + \langle \ddot{\gamma}, \gamma \rangle =0 \\
	&\Lr\ \langle \gamma, \ddot{\gamma} \rangle = - \langle \dot{\gamma}, \dot{\gamma} \rangle
\end{align*}
\\
Finalement l'EDP devient :
\[\ddot{\gamma} = - \langle \dot{\gamma}, \dot{\gamma} \rangle \gamma \]
\\
Or, il existe une paramétrisation de $\gamma$ telle que $\|\gamma\| = 1$. D'où les solutions :
\[\gamma(t) = \gamma(t_0) \cos (t - t_0) + \dot{\gamma}(t_0) \sin (t - t_0)\]





%\subsection{Algèbre et groupe de Lie} \label{ann:2Lie}

%\subsubsection{Quelques généraliés}

%\subsubsection{Cas particulier : $\bf{\U{1}}$}