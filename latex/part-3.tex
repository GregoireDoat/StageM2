
\section{Calcul pratique de la phase géométrique}

\subsection{}

\begin{itemize}
	\item L'idée c'est d'approcher la courbe par des géodésiques
	
	\item géodésique $\Lr\ \phaseg = \phaset$ donc c'est juste des arg !
	
	\item En cumulant le tout, ca fait donne l'invariant de Bargmann :
	
	\item Bonus : c'est super parce que, en pratique, on a toujours des données ponctuelles, dont pas de besoin de "choisir" la subtivision !
\end{itemize}

\subsection{Lien avec la première partie}

\begin{itemize}
	\item La présentation dans $\PC{n}$ à été fructueuse dimension (complexe) $n=1$, avec les paramètres de l'ellipse. Mais, au delà, ca devient moins clair. Intuitivement, on pourrait généraliser les états de polarisations en ajoutant simplement une matrice de rotation dans $\C^n$ au signal bivarié polarisé. Autrement dit, il faudrait découper $\PC{n}$ en une produit du genre :
	\[\PC{n} \cong \SO_n(\R)\times \PC{2}-ish\]
	C'est pas clair comment faire ça (est-ce que $\SU_n(\C)$ serait mieux ?)
\end{itemize}



\section{Application}

\subsection{Information apportée}

\begin{itemize}
	\item Si une source d'information n'est pas sensé changer d'état de polarisation, mais qu'on constate qu'il y a phase géométrique, alors ce n'est (quasiment) que du bruit. Ça peut donc être un premier filtre pour débruiter.
	\\
	
	Pour le corriger en revanche, c'est un peu plus technique parce que ça veut dire qu'il faut faut TUER la polarisation du signal.
	\\
	Une idée serait de regarder la moyenne du signal projeté dans $\PC{n}$ et du tout simplement remplacer l'état de polarisation du ce signal par ca moyenne.
	
	\item En étant un peu moins restrictif l'évolution de l'état de polarisation du signal, on a vu qu'un la phase instantanée doit contenir les hautes fréquences du signal, Peut-être qu'on pourrait se servir de ça pour contraindre les variations de l'état de polar et, là encore, en faire un critère de débruitage. Genre tué les hautes fréquences de la phase géométrique
	
	\item D'ailleurs, ca amène la question : QUIDE de faire du Fourier sur les phases ? La phase totale est pas très intéressante, mais les deux autres, elle sont, en gros à valeur dans $\R$.\\
	Donc qu'est-ce qu'on peut dire de $\fou{\phaseg}$ ? est-ce qu'on peut en faire des trucs (passe-bas notamment).
	
\end{itemize}




\subsection{Sensibilité de la phase géométrique par rapport au bruit}

\begin{itemize}
	\item Il faudrait voir comment la phase géométrique réagit au bruit.
	
	\item Si elle y est très résiliente, elle pourrait être source d'information sur des données. Voir même, être une information de base pour reconstruire une signal (genre en utilisant le fait qu'elle soit à propos des basse-fréquence)
	
	\item Plus généralement, il serait bien de voir comment elle réagit en fonction du spectre du bruit
	
\end{itemize}



\section{Cas non commutatif}

\subsection{Motivation}

\begin{itemize}
	
	\item Comparer les mesures de différents cite de mesure (chacun sont propre bruit, sa propre phase géo)
	
	\item Subspace tracking
	
\end{itemize}



\subsection{Possible généralisation des concepts abordées}

\begin{itemize}
	
	\item En gros on va devoir aller Stiefel, aka $\PC{n}$ va devenir une grassmannienne $Stiefel/\U{k}$
	
	\item Fubini-Study s'étend à ces espaces, et les grassmannienne sont toujours des variétés complexes
	
	\item La formule de la phase dyn va changer par contre (path ordering) et donc son interprétations sera à revoir :/
	
	\item
	
\end{itemize}