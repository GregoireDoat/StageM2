
\begin{itemize}
	\item On va donner trois exemples qui permettrons de voir à quoi ce peut servir, quelles sont les limites du notre modèle et comment les générliser en corps (et pourquoi)
	
	\item On va montrer crescendo dans les dimensions :\begin{itemize}
		
		\item en 2D  : pour faire le lien avec la première partie
		
		\item en 3D : ondes gravitationnelles générées par un système binaire. L'occasion de voir le problème d'interprétabilité du modèle avec le modèle et de parler de bruit
		
		\item en nD : système de capteur. Pour parler de ``l'état de polarisation'' (composante sur $\PC{n}$) même si le terme n'est plus approprié et du cas non commutatif
	\end{itemize}
\end{itemize}




\section{\todo Calcul pratique de la phase géométrique}

\begin{itemize}
	
	\item Il faut faire du calcul approximé d'aire sur $\PC{n}$. 2 pb : \begin{itemize}
		
		\item Calcul d'aire : MEH
		
		\item  $\PC{n}$ est une variété quotent : MEH
		
	\end{itemize}
	
	\item Solution : \begin{itemize}
		
		\item $\pi(\x) \cong \rho_{\x}$
		
		\item On simplifie tout par des approxes de géodésique
	\end{itemize}
	
	\item géodésique $\Lr\ \phaseg = \phaset$ donc c'est juste des arg ! (super fast)
	
	\item En cumulant le tout, ca fait donne l'invariant de Bargmann :
	
	\item Bonus : c'est super parce que, en pratique, on a toujours des données ponctuelles, dont pas de besoin de "choisir" les points de subdivision !
\end{itemize}




\section{\todo Exemples d'applications} \label{sec:exemples_appli}

\subsection{\todo Cas 2D : lien avec la première partie} \label{subsec:ex-2D}

\begin{itemize}
	
	\item $\PC{1} \cong S^2$
	
	\item Stokes donne exactement le calcul de la demi-aire \\
	
	\hrule
	
	\item Présentations des résultats de \cite{le_bihan_modephysiques_2023}
	
	\item \thoughts{je trouve qu'on s'éloigne un peu trop du sujet : $\phaseg$ pour les SIGNAUX. Là on a un EDP qui dirige le truc + j'ai rien à apporté dessus}
	
\end{itemize}



\subsection{\todo Cas 3D : application aux ondes gravitationnelles} \label{subsec:ex-3D}

\begin{itemize}
	
	\item Deux trous noirs tournent autour l'un de l'autre
	
	\item Ils sont caractérisés par 3 paramètres : masse, spin, l'autre
	
	\item 4 simulations : \begin{itemize}
		
		\item leur axe de rotations sont de moins en moins aligné
		
		\item ca affecte l'état de polarisation 
		
		\item ... et ca engendre de la phase géométrique qui se mesure
	\end{itemize}
	
	\item Cet exemple amène 2 questions : \begin{itemize}
		
		\item comment le modèle AM-FM-PM se généralise en hauteD ?
		
		\item comment la phase géométrique réagit au bruit
		
		\item on en reparle en dernière partie

	\end{itemize}
\end{itemize}

\subsection{\todo Cas nD : sytème de capteurs} \label{subsec:ex-nD}

\begin{itemize}
	
	\item cf \href{https://theses.hal.science/tel-00199884}{ici} pour les détails
	
	\item invariance par $\U{n}$ ? vraiment ?
\end{itemize}



\section{\todo Pour la suite}\label{sec:2lasuite}


\subsection{\todo Limite du modèle}\label{subsec:limite2model}

\begin{itemize}
	
	\item $\PC{1} \cong S^2$ est super satisfaisant mais au dela c'est beaucoup moins claire.
	
	\item Pour passer de 2D à 3D, on a juste augmenté les matrices de rotation, \ie~$\SO(3)$ au lieu de $\SO(2)$. 
	
	\item C'est ce qu'on fait par Lilly \& Olhede \cite{lilly_modulated_2011} et Lefèvre a discuté des généralisations dans sa thèse \cite{lefevre_polarization_2021}
	
	\item Ca fait sens physiquement mais par rapport à $\PC{n}$ c'est loin d'être évident
	
	\item ... est-ce que c'est là que se cache le lien avec le double-cover de $\SO$ par $\SU$ ???
	\begin{itemize}
		\item Il serait vraiment intéressant (instructif) de faire ce lien 
		
		\item peut-être notre travail sur le phase g est une généralisation de ça ? 
	\end{itemize}
	
\end{itemize}



\subsection{\todo Rapport de la phase géométrique au bruit}\label{subsec:rapportObruit}

\begin{itemize}
	\item Si une source d'information n'est pas sensé changer d'état de polarisation, mais qu'on constate qu'il y a phase géométrique, alors ce n'est (quasiment) que du bruit. Ça peut donc être un premier filtre pour débruiter.
	\begin{itemize}
		
		\item pour le corriger en revanche, c'est un peu plus technique parce que ça veut dire qu'il faut tuer la polarisation du signal ou au moins la réduire à une géodésique.
		
		\item une idée serait de regarder la moyenne du signal projeté dans $\PC{n}$ et du tout simplement remplacer l'état de polarisation du signal par ca moyenne. 
		
		\item ca devrait pas affecter le phase dynamique	
	\end{itemize}
	
	\item En étant un peu moins restrictif l'évolution de l'état de polarisation du signal, on a vu qu'un la phase instantanée doit contenir les hautes fréquences du signal, Peut-être qu'on pourrait se servir de ça pour contraindre les variations de l'état de polar et, là encore, en faire un critère de débruitage.
	\begin{itemize}
		
		\item  par exemple en tuant les hautes fréquences de la phase géométrique
		
		\item d'ailleurs, ca amène la question : quide de faire du Fourier sur les phases ? La phase totale est pas très intéressante (j'imagine), mais les deux autres ? 
		
		\item c'est un peu flou quand-même parce $\phaseg$ vient de l'état de polar, qui lui n'est affecté parles transfo de jauge
	\end{itemize}
	
	\hrule

	\item Il faudrait voir comment la phase géométrique réagit au bruit.
	\begin{itemize}
		
		\item Si elle y est très résiliente, elle pourrait être source d'information sur des données. Voir même, être une information de base pour reconstruire une signal (genre en utilisant le fait qu'elle soit à propos des basse-fréquence)
		
		\item Plus généralement, il serait bien de voir comment elle réagit en fonction de la distribution spectrale du bruit
	\end{itemize}
	
\end{itemize}



\subsection{\todo Cas non commutatif}\label{subsec:non-commu}

\begin{itemize}
	
	\item En gros on va devoir aller Stiefel, aka $\PC{n}$ va devenir une grassmannienne $Stiefel/\U{k}$
	
	\item Fubini-Study s'étend à ces espaces, et les grassmannienne sont toujours des variétés complexes
	
	\item La formule de la phase dyn va changer par contre (path ordering) et donc son interprétations sera à revoir :/
	
	\item Autres potentielles applications :
	\begin{itemize}
		
		\item Subspace tracking
		
		\item ?
	\end{itemize}
	
\end{itemize}

