\input{package&preset (nth)}


\begin{document}


\begin{titlepage}
	\centering
	{\color{white}l}\par
	
	\vspace{6.5cm}
	
	{\Large\textbf{\textsc{Titre}}}\par
	\vspace{1.5cm}
	{\large \textbf{Chapitre n : \textsc{Chapitre}}}\par
	
	\vfill
	
	{\large Grégoire \textsc{Doat}}\par
	
	\vspace{1cm}
	
\end{titlepage}


\setcounter{page}{1}
\tableofcontents


\part{C'est une Partie}


%\addcontentsline{toc}{section}{Introduction}

\section*{Introduction}
{\color{white}bllblblb}

bblblblblbl
\\ \\




\section{Section}

\subsection{Subsection}
\quad
oreigeuorgberbnerob \\
riuvbqireubqreiub
\begin{definition}
	eobiqhroghqerpoug
\end{definition}


\begin{enonce}[Voilà une Dinguerie]
	eobiqhroghqerpoug
\end{enonce}


\begin{lemme}
	contenu...
\end{lemme}


\begin{theoreme}
	zblblbl
\end{theoreme}


\begin{corollaire}
	Avec la démo :ieugzeugozebg \\
	jqpozrghfklvhzr \\ 
	qozrvizrovbrio
\end{corollaire}

\begin{demo}
Comme PNL
\end{demo}



\begin{remarque}
	Indented + italique
\end{remarque}



\begin{exercice}
	zblbl
	\begin{itemize}
		\item \begin{enumerate}
			\item eorigher
			\item zorgbqsdiuvb
		\end{enumerate}
	\end{itemize}
	\textperiodcentered\ zorifhsjlvnrin
\end{exercice}


\begin{annexe}
	\section{Annexes}

	\subsection{Une première annexe}
	
	\subsubsection{Avec sous-partie}
\end{annexe}



\newpage

\listoffigures
\vfill
\lstlistoflistings
\vfill
%\listtheorems

\newpage

\bibliography{../ref.bib}{}
\bibliographystyle{siam}
\end{document}