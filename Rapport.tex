
%%%%    IMPORTS 2 BASE    %%%%


\documentclass[hidelinks, french, oneside]{article}
\usepackage[utf8]{inputenc}
\usepackage[T1]{fontenc}

% pour le mise en page
\usepackage[a4paper, total={6.5in, 10in}]{geometry}
\usepackage{fontsize} \changefontsize[13]{10}		
\usepackage{xcolor}

% mathsymbole
\usepackage{amsmath, amssymb,stmaryrd}
\usepackage{wasysym}[mathcal] % a quoi sert le [mathcal] ?
\usepackage{nicefrac, units} % fraction pour text mode / unité

% TBD
\usepackage{mathcomp}
\usepackage{mathrsfs} % pour \mathscr a priori

% pour les belles fonts
\usepackage{amsfonts}
\DeclareMathAlphabet{\mathpzc}{OT1}{pzc}{m}{it}
%\usepackage{euscript}[mathcal]
%\usepackage{rsfso}
\usepackage{bbm}	 % mathbb étendu

% pour les hyprlien / cross-ref
\usepackage{hyperref, cleveref}
%\namecref{sec:label}

%pour les figures
\usepackage{graphicx} \usepackage{wrapfig, floatrow}
%\usepackage{subcaption} % surement useless, parsk floatrow-like 

% pour les beaux tableaux
\usepackage{multirow}

% pour des matrices infernales
\usepackage{easybmat}

% pour les citations (aucune idées de comment ca marche)
\usepackage{csquotes}



%%%%    RACCOURCIS    %%%%


% bb / cal / frak
\newcommand{\N}{\mathbb{N}}
\newcommand{\Z}{\mathbb{Z}}             % Note pour homo pour les updates
\newcommand{\Q}{\mathbb{Q}}
\newcommand{\R}{\mathbb{R}}
\newcommand{\C}{\mathbb{C}}
\newcommand{\K}{\mathbb{K}}
\renewcommand{\k}{\Bbbk}
\newcommand{\U}{\mathbb{U}}
\renewcommand{\u}{\text{U}}
\newcommand{\A}{\mathbb{A}}
\newcommand{\T}{\mathscr{T}}
\newcommand{\I}{\mathbb{I}}
\newcommand{\one}{\mathbbm{1}}
%\renewcommand{\S}{\mathfrak{S}}
\renewcommand{\S}{\mathbb{S}}

% arrows
\newcommand{\lr}{\longrightarrow}
\newcommand{\Lr}{\Longrightarrow}
%\renewcommand{\ll}{\longleftarrow}
\newcommand{\Ll}{\Longleftarrow}
\newcommand{\llr}{\longleftrightarrowr}
\newcommand{\Llr}{\Longleftrightarrow}

% espaces
\newcommand{\matk}{\mathpzc{M}_n(\mathbb{K})}
\newcommand{\matr}{\mathpzc{M}_n(\mathbb{R})}

% fonctions
\newcommand{\Arccos}{\text{Arccos}} 
\newcommand{\Arcsin}{\text{Arcsin}} 
\newcommand{\Arctan}{\text{Arctan}} 
\newcommand{\Argch}{\text{Argch}}       
\newcommand{\Argsh}{\text{Argsh}}

\newcommand{\congu}[1]{\overline{#1}}
\newcommand{\argmin}[1]{\underset{#1}{\text{argmin}}}
\newcommand{\argmax}[1]{\underset{#1}{\text{argmax}}}

\newcommand{\pgcd}{\text{pgcd}}
\newcommand{\PGCD}{\text{PGCD}}
\newcommand{\ppmc}{\text{ppcm}}
\newcommand{\sign}{\text{sign}}

\newcommand{\sgn}{\text{sgn}}
%\newcommand{\deg}{\text{deg}}
\newcommand{\ord}{\text{ord}}
\newcommand{\rot}{\text{rot}}

%\renewcommand{\det}{\text{det}}
\newcommand{\tr}{\text{tr}}
\newcommand{\rg}{\text{rg}}
\newcommand{\Co}{\text{com}}
\newcommand{\codim}{\text{codim}}

%espaces
%\renewcommand{\Vec}{\text{Vec}}
\newcommand{\im}{\text{Im}}
\newcommand{\Ker}{\text{Ker}}
\newcommand{\Ann}{\text{Ann}}
\newcommand{\Sp}{\text{Sp}} 
\newcommand{\GL}{\text{GL}}
\newcommand{\SL}{\text{SL}}
\newcommand{\SO}{\text{SO}}
\newcommand{\SU}{\text{SU}}
%\renewcommand{\div}{\text{div}}

\newcommand{\Aff}{\text{Aff}}
\newcommand{\HT}{\text{HT}}
\newcommand{\GA}{\text{GA}}

% spé proba
\newcommand{\esp}[2][]{\mathbb{E}_{#1}\left[\, #2\, \right]}
\newcommand{\var}[2][]{\mathbb{V}_{#1}\left[\, #2\, \right]}
%\newcommand{\Var}{\mathbb{V}}%\text{ar}}

% plus jolie
\renewcommand{\O}{\varnothing}
\renewcommand{\epsilon}{\varepsilon}
\renewcommand{\subsetneq}{\varsubsetneq}
\renewcommand{\leq}{\leqslant}
\renewcommand{\geq}{\geqslant}
\renewcommand{\limsup}{\varlimsup}
\renewcommand{\liminf}{\varliminf}

% autre
\newcommand{\defeq}{:=}
\renewcommand{\bf}[1]{\boldsymbol{#1}}
\renewcommand{\AC}{\sim}
\newcommand{\para}{\sslash}


% latin
\newcommand{\etal}{\textit{et al.}}
\newcommand{\etc}{\textit{etc.}}
\newcommand{\apriori}{\textit{a priori}}
\newcommand{\afortiori}{\textit{a fortiori}}
\newcommand{\acontrario}{\textit{a contrario}}
\newcommand{\infine}{\textit{in fine}}
\newcommand{\ie}{\textit{i.e.}}
\newcommand{\eg}{\textit{e.g.}}

% tmp
\newcommand{\phaset}{\Phi_{\text{tot}}}
\newcommand{\phased}{\Phi_{\text{dyn}}}
\newcommand{\phaseg}{\Phi_{\text{geo}}}



%%%%    ENONCES (PROP, DEF, RQ)    %%%%


\usepackage{amsthm}

% les styles
\newtheoremstyle{enonce}{0pt}{25pt}{}{}{\scshape}{\quad ---\quad }{0em}{}
\newtheoremstyle{special}{0pt}{25pt}{}{}{\scshape}{\quad ---\quad }{0em}{\thmnote{#3}}
\newtheoremstyle{rqlike}{0pt}{25pt}{\itshape}{}{\scshape}{\quad ---\quad}{0em}{}
\newtheoremstyle{exo}{0pt}{25pt}{\color{blue}}{}{\scshape\color{blue}}{: \newline}{0em}{}

% énoncés classiques
\theoremstyle{enonce}
\newtheorem{definition}{Définition}
\newtheorem{proposition}{Proposition}
\newtheorem{propriete}{Propriété}
\newtheorem{propricarac}[propriete]{Propriété Caractéristique}
\newtheorem{lemme}{Lemme}
\newtheorem{theoreme}{Théorème}[section]
\newtheorem{theodef}[theoreme]{Théorème et Définition}
\newtheorem{corollaire}{\qquad Corollaire}[theoreme]

% énoncé type
\theoremstyle{special}
\newtheorem{enonce}{}

% énoncé numérotation-less
\theoremstyle{rqlike}
\newtheorem*{remarque}{\qquad Remarque}
\newtheorem*{rappel}{\qquad Rappel}
\newtheorem*{exemple}{\qquad Exemple}

\theoremstyle{exo}
\newtheorem{exercice}{Exercice}

% pour cref aux enoncés
\crefname{definition}{définition}{Définition}
\crefname{proposition}{proposition}{Proposition}
\crefname{propriete}{propriété}{Propriété}
\crefname{lemme}{lemme}{Lemme}
\crefname{theoreme}{théorème}{Théorème}
\crefname{corollaire}{corollaire}{Corollaire} 

\crefname{remarque}{remarque}{Remarque}
\crefname{rappel}{rappel}{Rappel}
\crefname{exemple}{exemple}{Exemple}

\crefname{exercice}{exercice}{Exercice}

% espace démo  (pas convaincu)
\definecolor{mygray}{gray}{0.3}
\newtheoremstyle{demo}{8pt}{0pt}{\color{mygray}}{}{\itshape\color{mygray}}{\newline\newline}{0em}{}

\theoremstyle{demo}
\newtheorem*{demo}{\qquad\qquad\qquad\rule{3.5cm}{0.4pt}\qquad\quad Démonstration\qquad\quad \rule{3.5cm}{0.4pt}}
%\begin{center}\rule{8cm}{0.4pt}\end{center}



%%%%    PROG N PLOT    %%%%


% Prog

%\usepackage{fancyvrb} % pour le lore, mais useless AF
\usepackage{listings} 
\usepackage{pythontex}	% surement une dingz, à approfondire

% preset
\definecolor{codeblack}{gray}{0.2}
\definecolor{codegreen}{rgb}{0,0.6,0}
\definecolor{codegray}{gray}{0.5}
\definecolor{codepurple}{rgb}{0.58,0,0.82}
\definecolor{backcolour}{rgb}{0.98,0.98,0.95}

\lstdefinestyle{mystyle}{
backgroundcolor=\color{backcolour},   
commentstyle=\color{codegray},
keywordstyle=\color{orange},
numberstyle=\tiny\color{codegray},
stringstyle=\color{codegreen},
basicstyle=\color{codeblack}\ttfamily\footnotesize,
breakatwhitespace=false,         
breaklines=true,                 
captionpos=b,
abovecaptionskip=12,
belowcaptionskip=18,
keepspaces=true,                 
numbers=left,                    
numbersep=5pt,                  
showspaces=false,                
showstringspaces=false,
showtabs=false,                  
tabsize=2,
frame=single,
rulecolor=\color{lightgray}}
\lstset{style=mystyle}


% Plot

% pour dessins et les diagrams
\usepackage{tikz, tikz-cd, tkz-euclide}
\usepackage{scalerel, pict2e}
\usetikzlibrary{calc, patterns, shapes.arrows, arrows.meta, shadows, external} 

% pour les maxis plots
\usepackage{pgfplots}
\pgfplotsset{compat=newest, scaled y ticks=false}
\usepgfplotslibrary{groupplots, dateplot, statistics, fillbetween}

\tikzstyle{every node}=[font=\scriptsize]
\pgfplotsset{standard/.style={width=8cm,
	height=3cm,
	compat=1.18,
	trig format=rad,
	enlargelimits,
	axis x line=middle,
	axis y line=middle,
	enlarge x limits=0.15,
	enlarge y limits=0.15,
	every axis x label/.style={footnotesize, at={(current axis.right of origin)},anchor=north west},
	every axis y label/.style={footnotesize, at={(current axis.above origin)},anchor=south east},
	scale only axis=true}}
	
	
	
	%%%%    CAPTIONS 2 FIGURES    %%%%
	
	
	% set up des captions figures (extrêmement BG)
	
	% pos et séparateur générale
	\captionsetup{justification=centering}
	\DeclareCaptionLabelSeparator{custom}{\, ---\, }
	
	% spécial figure
	\DeclareCaptionLabelFormat{customfig}{\textit{fig. #2}}
	\DeclareCaptionFormat{customfig}{#1#2#3}
	%\DeclareCaptionFont{customfig}{\itshape} %marche pas jsppk
	\renewcommand{\thefigure}{\arabic{section}.\arabic{figure}}
	\captionsetup[figure]{labelformat=customfig, labelsep=custom}
	
	% spécial code (listings)
	\DeclareCaptionLabelFormat{customcode}{\textit{code \arabic{section}.#2}}
	\DeclareCaptionFormat{customcode}{#1#2#3}
	%\renewcommand{\thelstlisting}{\arabic{section}.\arabic{lstlisting}}
	\captionsetup[lstlisting]{labelformat=customcode, labelsep=custom}
	
	
	
	%%%%    TABLES/REF  &  HAUT/BAS DE PAGES    %%%%
	
	
	% nom des tables et références
	
	\renewcommand{\contentsname}{\begin{center}\textsc{Tables des Matrières}\end{center}}
	\renewcommand{\listfigurename}{\begin{center}\textsc{Table des Figures}\end{center}}
	\renewcommand{\lstlistlistingname}{\begin{center}\textsc{Table des Codes}\end{center}}
	\renewcommand{\refname}{\begin{center}\textsc{Références}\end{center}}
	
	% bas/haut de page (à optimiser à l'occas')
	
	\usepackage{fancyhdr}
	\pagestyle{fancy}                       
	\fancyhf{}
	\renewcommand{\headrulewidth}{0pt}
	\cfoot{\thepage}
	
	%version2réjane  :
	%\usepackage{eso-pic}
	%\renewcommand\headrulewidth{1pt}
	%\fancyhead[L]{\includegraphics[width=3cm]{logo_lr.png}}
	%\fancyhead[R]{\includegraphics[width=1cm]{logo_sdis17.png}}
	%\setlength{\headsep}{45pt}
	
	
	
	%%%%    SECTIONS ET TOC    %%%%
	
	
	% génération des sections
	
\usepackage[loadonly, toctitles, clearempty, newparttoc]{titlesec}

% liste des sections
\titleclass{\part}[0]{top} % 0 = niveau de section
\titleclass{\section}{straight}[\part]
\titleclass{\subsection}{straight}[\section]
\titleclass{\subsubsection}{straight}[\subsection]

% génère les numérotations avec format
%\newcounter{part}
\renewcommand{\thepart}{\Roman{part}}
%\newcounter{section}
\renewcommand{\thesection}{\Roman{section}}
%\newcounter{subsection}
\renewcommand{\thesubsection}{\arabic{section}.\arabic{subsection}}
%\newcounter{subsubsection}
\renewcommand{\thesubsubsection}{\arabic{section}.\arabic{subsection}.\arabic{subsubsection}}

% formatage
\titleformat{\part}[display]{\bfseries\scshape\Large}{\centering \rule{3.5cm}{0.4pt}\qquad Chapitre\quad \thepart \qquad \rule{3.5cm}{0.4pt}}{15pt}{\centering}[\vspace{0.2cm}\rule{8cm}{0.4pt}]
\titlespacing{\part}{0pt}{50pt}{70pt}
\newcommand{\partbreak}{\clearpage}

\titleformat{\section}{\bfseries\Large}{\thesection\quad ---}{15pt}{}
\titlespacing{\section}{10pt}{20pt}{15pt}
%\newcommand{\sectionbreak}{\clearpage}

\titleformat{\subsection}{\bfseries\large}{\thesubsection}{15pt}{}
\titlespacing{\subsection}{20pt}{15pt}{15pt}

\titleformat{\subsubsection}{\bfseries}{\thesubsubsection}{15pt}{}
\titlespacing{\subsubsection}{30pt}{15pt}{15pt}

% pour cref aux enoncés
\crefname{part}{partie}{Partie}
\crefname{section}{section}{Section}
\crefname{subsection}{sous-section}{Sous-section}


	% le TOC en lé-gende

\usepackage{titletoc}
%\contentsmargin{2em}

\titlecontents{part}[2em]{\bfseries\large\scshape 
	\hspace*{-1.5em}\rule{\textwidth}{0.5pt}\vspace{0.1cm}\\*
	 }{Chaptire\quad \contentslabel{-0.05em}\quad\ ---\quad}{\contentslabel{-1.5em} ---\quad }{\hfill\contentspage}[\hspace*{-0.5em}
\rule{\textwidth}{0.5pt}]

\titlecontents{section}[1.5em]{\addvspace{1em}\bfseries}{\contentslabel{1.25em} ---\quad}{\hspace*{-1.75em}}{\titlerule*[0.75pc]{.}\contentspage}[\addvspace{0.25em}]

\titlecontents{subsection}[3.8em]{\addvspace{0.15em}\normalfont}{\contentslabel{2em}}{\hspace*{-2em}}{\titlerule*[0.75pc]{.}\contentspage}

\titlecontents{subsubsection}[6.8em]{\normalfont}{\contentslabel{2.75em}}{\hspace*{-2.5em}}{\titlerule*[0.75pc]{.}\contentspage}


% set up d'un environnement pour les annexes

\newenvironment{annexe}{%
	\newpage
	
	% changement title sec
	\titleformat{\section}[display]{\bfseries\scshape\Large}{\centering}{15pt}{\centering}
	\titlespacing{\section}{0pt}{30pt}{40pt}
	
	\titleformat{\subsection}{\bfseries\large}{Annexe \thesubsection\quad ---\quad}{0pt}{}
	
	\titleformat{\subsubsection}{\bfseries}{\thesubsubsection}{15pt}{}
	
	% changement title toc
	\titlecontents{section}[0.25em]{\addvspace{0.5em}\bfseries}{}{\hspace*{-1.5em}}{\titlerule*[0.75pc]{.}\contentspage}
	
	\titlecontents{subsection}[1.5em]{\normalfont Annexe\hspace*{2.5em}}{\contentslabel{2em}}{\hspace*{-2em}}{\titlerule*[0.75pc]{.}\contentspage}
	
	\titlecontents{subsubsection}[6em]{\normalfont}{\contentslabel{2.75em}}{\hspace*{-2em}}{\titlerule*[0.75pc]{.}\contentspage}
	
	% changment de la numéritations
	\renewcommand{\thesubsection}{\Alph{subsection}}
	\renewcommand{\thesubsubsection}{\Alph{subsection}.\arabic{subsubsection}.}
	% mise à zéro des compters
	%\setcounter{section}{0}
}{}


\begin{document}

\begin{titlepage}
	%\AddToShipoutPictureBG*{\put(80,655){\includegraphics[width=2.9cm]{Logo MIX.png}}}
	%\AddToShipoutPictureBG*{\put(70,738){\includegraphics[width=5cm]{logo_lr.png}}}
	%\hspace{0.0cm} 
	%\AddToShipoutPictureBG*{\put(440,690){\includegraphics[width=3.0cm]{logo_sdis17.png}}}\\[5.0cm]
	%{\color{white}l}\par
	
	\centering
	\vspace{1.5cm}
	{\huge\textbf{Rapport de Stage de M2}}\par
	
	\vspace{2cm}
	{\huge\textbf{\textsc{Des trucs sûrement très cools}}}\par 
	\vspace{0.5cm}
	
	{\huge\textbf{\textsc{vraiment très très cool}}}\par
	\vspace{2.0cm}
	
	{\large Grégoire \textsc{Doat}}\par
	\vspace{0.5cm}
	\vfill
	
	% Bottom of the page
	{\large Encadré par Nicolas \textsc{Le Bihan},  Michel \textsc{Berthier}, \etal}\par
	\vspace{0.5cm}
	
	\rule{10cm}{0.4pt}\par
	\vspace{0.7cm}
	
	{Master \textsc{Mix} - Université de La Rochelle}\par
	\vspace{0.25cm}
	
	{\large 2024 - 2025}
\end{titlepage}

\newpage
\tableofcontents
\thispagestyle{empty}
{\color{white}l}


\newpage
\setcounter{page}{1}

\phantomsection
\addcontentsline{toc}{section}{Introduction}
\section*{Introduction}

Quote : The geometric phase is also one of the most beautiful examples of what Wigner once called ``the unreasonable effectiveness of mathematics in the natural sciences.'' \cite[p.\,4]{bohm_geometric_2003}

\begin{itemize}
	
	\item Pourquoi on va dans les complexes ? Réponse \cref{sec:temp-freq}
	
	\item Pourquoi Schrödinger ? Peut-être que Mukunda \cite{mukunda_quantum_1993} à la réponse
	
	\item En quoi la phase géométrique est une phase ? lien avec Fourier ?
	
	\item Lien entre phase et polarisation : différences, intérêts ? C'est une phase dans la polar IMO
	
	\item La param $(a,\chi,\theta,\varphi)$ elle dit quoi vraiment ? (sûrement simple à généraliser btw) + $(\chi,\theta)$ est pas en bijection avec la sphère (surement osef par contre)
	
	\item Multivaluation dans les variétés \cite[p. 8]{bohm_geometric_2003} ???
	
\end{itemize}



\part{Préliminaire}





%\section{Signal analytique, transformé de Hilberts, Analyse en temps-fréquence}
\section{Sur l'intérêt de transformé les signaux réels en complexes}\label{sec:temp-freq}


\begin{enonce}[Intro de la \nameCref{sec:temp-freq}]
	\textit{Malgré ce que l'énoncé de la problématique laisse entendre, dans toute la suite nous travaillerons avec des signaux à valeur de $\R$ dans $\C^n$ plutôt que dans $\R^n$.
		Outre le plaisir coupable des mathématiciens à vouloir généraliser les choses, il y a de très bonne raison à la transformation des signaux réels en complexes et c'est sera l'objet de cette première  \namecref{sec:temp-freq}.}
\end{enonce}

\textit{Cette partie reprends, dans les grandes lignes, les propos de \textsc{Cohen} dans son livre \emph{Time frequency analysis} \cite{cohen_time_1995}}
Dans l'étude de signaux, la transformée de Fourier est un outil standard puisqu'il donne accès à tout l'information fréquentielle de ce dernier. Ce gain d'information n'est pas gratuit pour autant puisqu'on perd avec tout notion de temporalité. Pourtant dans bien des cas, une information instantanée, dépendante en temps, est plus pertinente.
\\
C'est par exemple le cas dans l'étude de la langue oral (bon terme ?). Le sens d'une phrase ne vient pas du signal, \ie~la voix, tout entier mais plutôt de ses valeurs locales. Lorsque l'on prononce le mot ``\textit{fleur}'', c'est l’enchaînement des sons associés au ``\textit{f}'', ``\textit{l}'', ``\textit{eu}'', \etc~qui est important et non la structure global du signal.  
\\
On pourrait également cité l'effet Döppler qui permet, entre autre, de savoir si un émetteur s'éloigne ou se rapproche. Dans ce cadre, le passage d'un signal de hautes à basses fréquences (typiquement d'un son aigu à un son grave) indique que la source, qui se rapprochait, s'est mise à s'éloigner : c'est la variation de fréquence en cours du temps qui est porteuse d'information.
\\

Pour avoir une notion de fréquence instantanée, il serait utile de pouvoir écrire tout signal réel $x$ sous la forme :
\begin{equation}\label{eq:amp-phase_instant}
	x(t) = a(t) \cos\phi(t)
\end{equation}
\\
où $a$ correspondrait à l'\textit{amplitude instantanée} du signal et $\phi$ sa \textit{phase instantanée}. Le problème d'une telle décomposition est que, si elle existe bel et bien, elle n'est en revanche pas unique. L'exemple le plus simple étant le cas $\ x(t) = \sin(t)\ $ qui se représente, entre autre, par les pairs :
\begin{align*}
	\big(a(t),\phi(t)\big) &= \big(1, t+\nicefrac{\pi}{2}\big)  &  
	\big(a(t),\phi(t)\big) &= \big(\sin(t), 0\big)  &  
	\big(a(t),\phi(t)\big) &= \big(2\sin(\nicefrac{t}{2}), \nicefrac{t}{2}\big)
\end{align*}
{\color{white}l}

Pour avoir unicité de cette décomposition, il nous faut donc une contrainte sur $(a,\phi)$. Une approche serait de voir $x$ comme la partie réelle du signal complexe :
\[\forall t\in\R,\ z_x(t) = a(t)e^{i\phi(t)}\quad \Lr\quad x(t) = \Re e\, z_x(t) = a(t) \cos \phi(t)\]
\\
Dans ce cas on aurait bien unicité de $a$ et $\phi$ par rapport à $z_x$ (son amplitude et sa phase) mais cela ne fait que déplacer le problème puisque $z_x$ n'est pas mieux défini : Il y a une liberté totale quant au choix sa partie imaginaire. Pour motiver la définition de $z_x$, sont rappeler quelques outils d'analyse temps-fréquence.



\subsection{Distribution de l'énergie en temps et fréquence}\label{subsec:distrib_temp-freq}

Dans toute cette \namecref{sec:temp-freq}, on considèrera $x$ un signal complexe et on notera $\hat{x}$ ou $\mathcal{F}(x)$ sa transformée de Fourier (dont on supposera quelle existe, typiquement $x\in L^1(\R, \C)$) :
\begin{align*}
	x\ &:\quad \begin{aligned}\R\ &\lr\quad \C \\ x\ &\longmapsto\ x(t)
	\end{aligned}  &  \mathcal{F}(x)=\hat{x}\ &:\quad \begin{aligned}\R\ &\lr\qquad\quad \C \\ \nu\ &\longmapsto\ \int_\R x(t)e^{2\pi i \nu t}dt
	\end{aligned}
\end{align*}
\\

\begin{definition}[Densités d'énergie]\label{def:densite_dE}
	Étant donnée un signal complexe $x$, la \emph{densité d'énergie} $\rho$ de $x$ est donnée par le carré de son module. De même on définit $\varrho$ la \emph{densité d'énergie spectrale} :
	\begin{align*}
		\rho\ &:\quad \begin{aligned}\R\ &\lr\quad \R^+ \\ t\ &\longmapsto\ \big|x(t)\big|^2 \end{aligned}  &
		\varrho\ &:\quad \begin{aligned}\R\ &\lr\quad \R^+ \\ \nu\ &\longmapsto\ \big|\hat{x}(t)\big|^2 \end{aligned}
	\end{align*}
	\\
	La valeur $\rho(t)$ correspond à la puissance déployée pour émettre le signal à l'instant $t$ et $\varrho(\nu)$ à l'énergie associée à la fréquence $\nu$ sur tout le signal. 
\end{definition}

Par exemple, si $\ x(t)=e^{2\pi i\nu_0 t}$, alors $\ \hat{x}(t) = \delta(x-\nu_0)$. Dans ce cas, on a les densités :
\begin{align*}
	\rho(t) &= 1  &  \varrho(\nu) = \delta(\nu-\nu_0)
\end{align*}
Du point de vu temporel, le signal est émis avec une puissance régulière, mais le fait que $\varrho$ soit un dirac indique que toute l'énergie du signal est concentré dans une unique fréquence.
\\ \\

\begin{propriete}[\'Egalité de Parceval]\label{propri:parceval}
	La transformée étant une isométrie de l'espace $L^2(\R,\C)$, si $x\in L^2(\R,\C)$, alors l'énergie totale du signal est indépendante de la représentation de ce dernier (temporelle ou spectrale) :
	\begin{equation}
		E(x) \defeq {\|x\|_2}^2 = \int_\R \rho(t) dt = \int_\R \varrho(\nu) d\nu = {\|\hat{x}\|_2}^2
	\end{equation}	
\end{propriete}

\begin{definition}[Durée et largeur de bande]\label{def:band-width}
	L'espérance et la variance de ces densités sont notées (pour peu qu'elles existent) :
	\begin{align*}
		\esp[\rho]{t} &\defeq \int_\R t \big|x(t)\big|^2 dt   &  \var[\rho]{t} &\defeq \esp[\rho]{\big(t-\esp[\rho]{t}\big)^2} = \esp[\rho]{t^2} - \esp[\rho]{t}^2  \\
		\esp[\varrho]{\nu} &\defeq \int_\R \nu \big|\hat{x}(\nu)\big|^2 d\nu  &  \var[\varrho]{\nu} &\defeq \esp[\varrho]{\big(\nu - \esp[\varrho]{\nu}\big)^2} = \esp[\varrho]{\nu^2} - \esp[\varrho]{\nu}^2
	\end{align*}
	\\
	Si un signal est très localisé temporellement, alors la première espérance donne une idée de l'instant d'émission du signal. Si \acontrario, le signal est localisé en fréquence, la seconde espérance peut s'interpréter comme la fréquance moyenne ou fréquence ''dominante'' dans le signal. \\
	Hors mis ces cas, $\esp[\rho]{t}$ et $\esp[\varrho]{\nu}$ ne sont difficilement interprétables. En particulier, et ce sera important pour la suite, dans le cas des signaux réels, l'espérance de $\varrho$ est toujours nulles.
	
	Les écart-types, en revanche, sont plus facilement interprétable : Le premier est appelé \emph{durée d'émission} du signal puisqu'il renseigne l'étalement temporelle du signal. Le second, est appelé \emph{largeur de bande} puisque, lui, renseigne l'étalement fréquentielle du signal. 
	\\
	\`A noter que si le support du signal $x$ n'est pas connexe, alors la durée d'émission donnera plutôt le différentiel entre la première période d'émission et la dernière.
\end{definition}

\begin{proposition}[Integration trick, 1.4 de \cite{cohen_time_1995}] \label{prop:integ_trick}
	Si le signal est $n$ fois dérivable et que la densité d’énergie associée $\varrho$ admet un moment d'ordre $n$, alors il est donnée par la formule :
	\begin{equation}\label{eq:moment_f}
		\forall n\in\N,\qquad \esp[\varrho]{\nu^n} = \left(\frac{i}{2 \pi}\right)^n  \int_\R x(t) \frac{d^n}{dt^n} \overline{x(t)} dt = \left(\frac{i}{2 \pi}\right)^n  \left\langle x, \frac{d^n}{dt^n}x\right\rangle
	\end{equation}
\\
Avec les hypotèses analogues, les moments $\rho$ s'écrivent :
\begin{equation}\label{eq:moment_t}
	\forall n\in\N,\qquad \esp[\rho]{t^n} = \left(\frac{1}{2i \pi}\right)^n  \int_\R \hat{x}(\nu) \frac{d^n}{dt^n} \overline{\hat{x}(\nu)} dt = \left(\frac{1}{2i \pi}\right)^n  \left\langle \hat{x}, \frac{d^n}{d\nu^n}\hat{x}\right\rangle
\end{equation}

\begin{demo}
	\`A supposer que les intégrales existes et que le théorème de Fubini s'applique, on a $\forall n\in\N$ :
	\begin{align*}
	\esp[\varrho]{\nu^n} = \int_\R \nu^n\varrho(\nu) d\nu &= \int_\R \nu^n \hat{x}(\nu)\overline{\hat{x}(\nu)} d\nu \\
		&= \int_\R \nu^n \int_\R x(t)e^{-2i \pi \nu t} dt \int_\R \overline{x(t')}e^{2i \pi \nu t'} dt' d\nu \\
		&= \int_\R \int_\R x(t) \overline{x(t')} \int_\R \nu^n e^{-2i \pi \nu (t-t')} d\nu dt dt' 
	\end{align*}
	Ici, on remarque que :
	\begin{align*}
	\nu^n e^{-2i \pi \nu (t-t')} &= \nu^{n-1}\frac{1}{-2i \pi}\frac{d}{dt}e^{-2i \pi \nu(t-t')} \\
		&= \nu^{n-2}\frac{1}{(-2i \pi)^2}\frac{d^2}{dt^2}e^{-2i \pi \nu(t-t')} \\
		 &\ \vdots \\
		&= \frac{1}{(-2i \pi)^n}\frac{d^n}{dt^n}e^{-2i \pi \nu(t-t')}
	\end{align*}
	Ce qui permet, en jouant sur les ordres d'intégrations, d'obtenir :
	\begin{align*}
	\esp[\varrho]{\nu^n} &= \int_\R \int_\R x(t) \overline{x(t')} \int_\R \nu^n e^{-2i \pi \nu (t-t')} d\nu\, dt\, dt' \\
		&= \int_\R \int_\R x(t) \overline{x(t')} \int_\R \frac{1}{(-2i \pi)^n}\frac{d^n}{dt^n}e^{-2i \pi \nu(t-t')} d\nu\, dt\, dt' \\
		&= \frac{1}{(-2i \pi)^n} \int_\R \int_\R x(t) \overline{x(t')} \frac{d^n}{dt^n}\int_\R e^{-2i \pi \nu(t-t')} d\nu\, dt\, dt' \\
		&= \left(\frac{1}{-2i \pi}\right)^n \int_\R \int_\R x(t) \overline{x(t')} \frac{d^n}{dt^n}\mathcal{F}\big(1\big)(t-t') dt\, dt' \\
		&= \left(\frac{i}{2 \pi}\right)^n \int_\R \int_\R x(t) \overline{x(t')} \frac{d^n}{dt^n}\delta(t-t') dt\, dt' \\
		&= \left(\frac{i}{2 \pi}\right)^n \int_\R x(t) \int_\R \overline{x(t')} \frac{d^n}{dt^n}\delta(t-t') dt' dt \\
		&= \left(\frac{i}{2 \pi}\right)^n  \int_\R x(t) \frac{d^n}{dt^n}  \overline{x(t)} dt
	\end{align*}
\end{demo}
\end{proposition}

\begin{enonce}[Fréquence Instantanée]
	On écrit le signal et sa transformée sous forme exponentielle, de sorte que :
	\begin{align*}
		x(t) &= a(t)e^{i\phi(t)}  &  \hat{x}(\nu) &= \alpha(t)e^{i\psi(t)}
	\end{align*}

\begin{itemize}
	
	\item Déjà, pour un signal classique, $\cos(\nu t+\phi)$, on considère $\phi$ la phase et $\nu$ la fréquence. En version complexe, pour le signal $e^{i(\nu t + \phi)}$, c'est tout $\nu t+\phi$ la phase et la fréquence est donnée par la dérivée de la phase :
	\[\nu = \frac{d}{dt}(\nu t+\phi)\]
	
	\item On peut voir la dérivée de la phase comme la fréquence instantanée parce que intégré du le temps, la donne la fréquence moyenne :
	\[\esp[\varrho]{\nu} = \int_\R \phi'(t)\rho(t)dt\]
	
	\item Le carré de la largeur de bande de $x$ s'écrit :
	\begin{equation*}\label{eq:bandwidth_cohen}
		\begin{aligned}
			B^2  = \var[\varrho]{\nu} &= \int \left( \frac{a'(t)}{a(t)}\right)^2 a^2(t) dt + \int \Big(\phi'(t) - \esp[\varrho]{\nu}\Big) a^2(t) dt \\
			&= \qquad \esp[\rho]{\left( \frac{a'}{a}\right)^2} \qquad\ \, + \qquad\ \, \esp[\rho]{\phi' - \esp[\varrho]{\nu}} \\
			&= \qquad\quad {B_{AM}}^2 \qquad\ \ + \qquad\qquad\ {B_{FM}}^2
		\end{aligned} \qquad\qquad\text{ eq. (1.96) \cite{cohen_time_1995}}
	\end{equation*}
	On a séparation nette entre la phase (la fréquence vraiment) et l'amplitude dans la largeur de bande !\\
	+ ca renforce l'idée que $\phi'$ correspond la fréquence instantanée \\
	+ pythagore, qu'est-ce qu'il fout là
\end{itemize}
\end{enonce}


\subsection{Problème de signaux réel}


\begin{itemize}
	\item Le signal $x$ étant réel, son spectre est à symétrie hermitienne. En notant $\hat{x}$ la transformé de Fourier de $x$, on a :
	\[\forall t\in\R,\ x(t)\in\R \quad \Lr \quad \forall \nu\in\R,\ \hat{x}(-\nu) = \congu{\hat{x}(\nu)}\]
	
	\item Redondance d'info + pas physiquement intéressant les fréquences négatives
	
	\item Si on veut la fréquence moyenne associé à $x$ (\ie~$\esp[\varrho]{\nu}$), alors nécessairement elle sera nulle (intégrale impair) : pas intéressant + la largeur de bande dit plus rien non plus du coup :/
	
	\item La covariance de signaux réel est nulle (pas informatif + on aimerait que ca le soit). Formule de la covariance :
	\[\text{Cov}(tw) = \int t \phi'(t) \big|x(t)\big|^2 dt - \int t \big|x(t)\big|^2\int \nu \big|\hat{x}(\nu)\big|^2 d\nu\]
\end{itemize}

\subsection{Signal Analytique}

\begin{itemize}
	\item les contraintes : 
	\begin{equation}
		\Re e\ \big(\mathcal{A}x\big)(t) = x(t)  \qquad\qquad\text{ et } \qquad\qquad  \widehat{\mathcal{A}x}(\nu) = 2\one_{\R^+}(\nu)\hat{x}(\nu)
	\end{equation} 
	parce que y'a de la redondances anyway sur $|\hat{x}|$.
	
	\item amène à la transfo de Hilberts (\href{https://les-mathematiques.net/vanilla/discussion/859485/transformee-de-fourier-de-heaviside}{\color{blue}\underline{fourier de Heavyside propre}})

	\item ATTENTION : si on pose $x(t)=\alpha(t)\cos\varphi(t)$, rien n'indique qu'on aura $(\alpha,\varphi) = (a,\phi)$ ! si c'est pas le cas, c'est le décomposition n'était "pas la bonne". C'est bizarre, mais c'est comme ca !
		
	\item un mot sur la fréquence instantané.
	
	\item (lien avec la transformée en ondelette et pourquoi on fait pas de ça ?)

\end{itemize}

\subsection{Cas dans signaux AM-FM}


\section{Réflexion autour du produit hermitien}


Soit $x,y\in\C^n$ des vecteurs complexes et $X,Y\in\R^{2\times n}$ leur versions réelles. On note $x^j$ sa $j^{eme}$ composante complèxe et $x_1$ (resp. $x_2$) le vecteur composé de ses parties réelles (resp. imaginaires) :
\[x = \big(x^j\big)_j =  x_1 + ix_2 =  \big(x^j_1\big)_j +i \big(x^j_2\big)_j\]
\\
On a deux façon d'écrire le produit hermitien (canonique) de $X$ avec $Y$ :
\begin{align*}
	\langle x,y \rangle = \langle x_1 + ix_2, y_1 + iy_2\rangle &= \langle x_1, y_1\rangle - i \langle x_1,y_2\rangle +i\langle x_2, y_1\rangle + \langle x_2, y_2\rangle  \\
	&= \langle x_1, y_1\rangle + \langle x_2, y_2\rangle 
	+ i\big(\langle x_2, y_1\rangle - \langle x_1,y_2\rangle\big) \\
	&= \sum_j x^j_1 y^j_1+ x^j_2 y^j_2
	+ i\left(\sum_j x^j_2 y^j_1 -  x^j_1y^j_2\right) \\
	&= \left\langle \begin{pmatrix} x_1 \\ x_2 \end{pmatrix},\begin{pmatrix} y_1 \\ y_2 \end{pmatrix}\right\rangle
	+ i\left\langle \begin{pmatrix} x_1 \\ x_2 \end{pmatrix},\begin{pmatrix} -y_2 \\ y_1 \end{pmatrix}\right\rangle \\
	&= \Big\langle X,Y\Big\rangle 
	+ i\left\langle X,\begin{pmatrix} 0 & -I_n \\ I_n & 0 \end{pmatrix}\begin{pmatrix} y_1 \\ y_2 \end{pmatrix}\right\rangle\\
	&= \Big\langle X,Y\Big\rangle 
	+ i\left\langle X,\begin{pmatrix} 0 & -I_n \\ I_n & 0 \end{pmatrix}Y\right\rangle
\end{align*}
\\
Cette formule peut s’interpréter en disant que le produit hermitien encode le produit scalaire entre $X$ et $Y$ et le produit scalaire de $X$ avec les vecteurs $y^j=(y^j_1, y^j_2)$  auquel on aurait applique une rotation de $90^\circ$ (rotation qui, par ailleurs, correspond à la multiplication par $i$ dans le plan complexe). Moralement, $\langle x,y \rangle =0$ demande une orthogonalité de $X$ à un plan, ce qui fait sens puisque cela tient compte du fait que les $x^j, y^j$ sont complexes (donc de dimension 2 en tant que $\R-$e.v.).
\\
On a aussi l'écriture (quand-même moins clair) :
\begin{align*}
	\langle x,y \rangle &= \langle x_1, y_1\rangle + \langle x_2, y_2\rangle 
	+ i\big(\langle x_2, y_1\rangle - \langle x_1,y_2\rangle\big) \\
	&= \sum_j x^j_1 y^j_1+ x^j_2 y^j_2+ i\sum_j \big( x^j_2 y^j_1 - x^j_1y^j_2 \big) \\
	&= \sum_j \big\langle X^j,Y^j\big\rangle - i\sum_j \det(X^j, Y^j)
	%&= \sum_j \|X^j\|\|Y^j\| \cos \widehat{X^j,X^j} + i\sin \widehat{X^j,Y^j}
\end{align*}
Cette formule dit que les parties reélles et imaginaires du produit $\langle x,y \rangle$ encodent respectivement ``l'orthogonalité moyenne'' et la ``linéarité moyenne ''entre les familles de vecteurs $X^j\in\R^2$ et $Y^j\in\R^2$. L'orthogonalité d'une part parce que le produit scalaire s'annule en cas d'orthogonalité (no shit), la linéarité d'autre part car le déterminant s'annule en cas de colinéarité et moyenne car se sont des sommes sur $j$. \textbf{$\bf{\langle x,y \rangle=0}$ ne dit pas que les le vecteurs sont à la fois colinéaire et orthogonaux parce que ce sont des valeurs moyennes (\ie annuler une somme ne veut pas dire que chacun des termes sont nuls).}
\\

Si maintenant on s'intéresse au cas $y=x$, on a $\forall h\in\C^n$ :
\begin{align*}
	\langle x+h, x+h \rangle = \langle x, x \rangle + \langle x, h \rangle + \langle h, x \rangle + \langle h, h \rangle 
	&= \langle x, x \rangle + \langle x, h \rangle  + \overline{\langle x, h \rangle }+ \langle h, h \rangle \\
	&= \langle x, x \rangle + 2\Re e \langle x, h \rangle + \langle h, h \rangle
\end{align*}
Donc si $x\in\C^n$ est fonction d'un paramètre $t$, l'égalité $\ \langle x, \dot{x} \rangle = \frac{1}{2}\partial_t\langle x, x \rangle\ $ du cas réel devient :
\[\langle x, \dot{x} \rangle = \frac{1}{2}\partial_t\langle x, x \rangle + i\left\langle X,\begin{pmatrix} 0 & -I_n \\ I_n & 0 \end{pmatrix}\dot{X}\right\rangle\]
\\
En particulier, quand bien-même $x$ serait de norme constante, on aurait toujours un degré de liberté pour $\ \langle x, \dot{x} \rangle$ :
\[\|x\|=c\quad \Lr\quad \langle x, \dot{x} \rangle = i\left\langle X,\begin{pmatrix} 0 & -I_n \\ I_n & 0 \end{pmatrix}\dot{X}\right\rangle\]
\\

\begin{definition}\label{def:phase_dyn}
	Ainsi, il reste tout un degré de liberté au produit $\ \langle x, \dot{x} \rangle\ $ même si $x\in\S^{2n}$. En intégrant ce degré de liberté supplémentaire, c'est-à-dire en tenant compte de son évolution sur la période $[t_0,t]$, l'on obtient ce qui est appeller le \emph{phase dynamique} :
	\[\phased \defeq \phased(t_0,t) = \int_{t_0}^t \Im m \big\langle \psi(s) \, |\, \dot{\psi}(s) \big\rangle ds\]
	Elle dynamique en cela qu'elle est propre au variation de $\psi$ et qu'elle considère tout l'évolution de $\psi$ : ça dynamique.
	\textit{Rq} : Avec les notations de la \cref{sec:temp-freq} et si le signal est d'amplitude constante, alors on a :
	 \[\phased(-\infty, +\infty) = \int_\R \big\langle \psi(s) \, |\, \dot{\psi}(s) \big\rangle ds = -2\pi\int_\R \nu\big|\hat{\psi}(\nu)\big|^2 d\nu = - 2\pi \esp[|\hat{\psi}|^2]{\nu}\]
	
	\textbf{... et pourquoi c'est une PHASE du coup ?}
\end{definition}


\begin{definition}[Connexion de Berry]\label{def:berry_connx}
	On appelle \emph{connexion de Berry} le champ de forme linéaire :
	\begin{equation}\label{eq:berry_connx}
		\forall \psi\in\mathpzc{M},\quad A_\psi :\ \begin{aligned} T_\psi\mathpzc{M}\ &\lr\qquad\ \R \\ \phi\quad &\longmapsto\ \Im m \big\langle \psi(s) \, |\, \phi(s) \big\rangle
		\end{aligned}
	\end{equation}
	\textbf{Elle a rien d'une connexion par contre :/}
\end{definition}





\setcounter{section}{0}
\setcounter{figure}{0}
\setcounter{lstlisting}{0}

\part{Phase Géométrique}

\section{Description des signaux multivariés}\label{sec:bases}

\subsection{Version Lilly \cite{lilly_bivariate_2010, lilly_analysis_2012}}

On a un signal bivarié $\bf{x}(t) = \big(x(t),y(t)\big)$ qu'on transforme (voir \cref{subsec:SA_Hilb}) soit la forme :
\[z_x(t) = \begin{pmatrix}x_+(t) \\ y_+(t)\end{pmatrix} = \begin{pmatrix}a_x(t) e^{i\phi_x(t)} \\ a_y(t) e^{i\phi_y(t)}\end{pmatrix}\in\C^2\]
\\

\`A côté de ça, on a les ellipses modulées :
\[z(t) = e^{i\theta}\big(a(t)\cos\phi(t) + ib(t) \sin\phi(t)\big) = A(t) e^{i\theta} \big( \sin\chi(t) \cos\phi(t) + i\sin\chi(t) \sin\phi(t) \big) \in\C\]
Qui sous forme vectoriel se réécrit :
\begin{equation}\label{eq:exp_elliptik}
	z(t) = e^{i\phi} R_{\theta(t)}\begin{pmatrix} a(t) \\ -ib(t) \end{pmatrix} = A(t)e^{i\phi} R_{\theta(t)} \begin{pmatrix} \cos\chi(t) \\ -i\sin\chi(t) \end{pmatrix} \in\C^2,\qquad R_\theta\in\SO_2(\R) \
\end{equation}
\\

Pour avoir la désinscription de $\bf{x}$ en terme d’ellipse, il suffit donc de poser :\footnote{C'est la version analytique du la version vectorielle de l'ellipse !}
\[z_x(t) = z(t)\quad \Llr\quad \begin{pmatrix}a_x(t) e^{i\phi_x(t)} \\ a_y(t) e^ {i\phi_y(t)}\end{pmatrix} = A(t)e^{i\phi} R_{\theta(t)} \begin{pmatrix} \cos\chi(t) \\ -i\sin\chi(t) \end{pmatrix}\]
\\
Ensuite, on pose :
\[\begin{pmatrix}z_+ \\ z_-\end{pmatrix} = \begin{pmatrix}a_+ e^{i\phi_+} \\ a_- e^{i\phi_-}\end{pmatrix} = \frac{1}{2}\begin{pmatrix}x_+ + iy_+ \\ x_+ - iy_+\end{pmatrix} = \frac{1}{2}\begin{pmatrix}1 & i \\ 1 & -i\end{pmatrix} \begin{pmatrix}x_+ \\ y_+\end{pmatrix}\]
\\
Et on a :
\begin{align*}
	2\phi &= \phi_+ + \phi_-  &  a &= A\cos\chi = a_+ + a_- \\
	2\theta &= \phi_+ - \phi_-  &  b &= A\sin\chi = a_+ - a_- 
\end{align*}
et on en déduit :
\begin{align*}
	A &= \sqrt{(a_+ + a_-)^2 + (a_+ - a_-)^2}  &  \begin{aligned} \cos\chi &= \frac{a_+ + a_- }{\sqrt{(a_+ + a_-)^2 + (a_+ - a_-)^2}}  \\  \sin\chi &= \frac{a_+ - a_- }{\sqrt{(a_+ + a_-)^2 + (a_+ - a_-)^2}}	\end{aligned}
\end{align*}
Ce qui donne \infine :
\[\begin{pmatrix}x_+ \\ y_+\end{pmatrix} = e^{i\frac{\phi_+ + \phi_-}{2}} R_{\frac{\phi_+ - \phi_-}{2}} \begin{pmatrix}a_+ + a_- \\ -i(a_+ - a_-)\end{pmatrix}\]
\\

L'\cref{eq:exp_elliptik} ce généralise  très bien, il suffit d'augmenter la taille de $R_\theta\in\SO_n(\R)$ et de lui donner le vecteur étendu :\footnote{\textit{Sachant que le vecteur contenant $a$ et $b$ est principalement nul, on peut réécrire le produit ne considérant que les deux premières colonnes de $R_\theta$.}}
\[z_x(t) = \begin{pmatrix}x_{1+}(t) \\ \vdots \\ 
	x_{n+}(t)\end{pmatrix} = e^{i\phi} R_{\theta(t)}\begin{pmatrix} a(t) \\ -ib(t) \\ 0 \\ \vdots \\ 0 \end{pmatrix} = A(t)e^{i\phi} R_{\theta(t)} \begin{pmatrix} \cos\chi(t) \\ -i\sin\chi(t) \\ 0 \\ \vdots \\ 0 \end{pmatrix}\]
\\

Maintenant, la question est de savoir comment généraliser la transformation en $(z_+, z_-)$ pour obtenir les paramètres $(A, \phi, R_\theta, \chi)$ dans ce cas...
\\
Pour généraliser le procédé, on peut noter que :
\[\begin{pmatrix}z_+ \\ z_-\end{pmatrix} = \frac{1}{2}\begin{pmatrix}1 & i \\ 1 & -i\end{pmatrix} \begin{pmatrix}x_+ \\ y_+\end{pmatrix} = \frac{1}{\sqrt{2}}U \begin{pmatrix}x_+ \\ y_+\end{pmatrix}\qquad\qquad \text{avec }\ U=\frac{1}{\sqrt{2}}\begin{pmatrix}1 & i \\ 1 & -i\end{pmatrix}\in\U(2)\]
\\ 
Ce qui ramène à se demander comment généraliser $U$ à $\SU(n)$. Le problème est que $U$ est indépendant de tout les paramètres $(A, \phi, R_\theta, \chi)$ et sa généralisation est vraiment pas évidente sachant qu'on que le formule avec $n=2$... et pour $n=3$ ca devient déjà chaud (pour rappelle $\dim \SO_n(\R)=\frac{n(n-1)}{2}$ et donc $\theta\in\R^n$, ce qui rend le problème de pire en pire à mesure qu'on augmente $n$).



\subsection{Mon blabla}\label{subsec:blabla}


\begin{definition}[Signal multivarié] \label{def:signal_multivar}
Un \emph{signal multivarié}, ou \emph{$n-$varié}, signal à valeur dans $\C^n$. Formellement, c'est une fonction de carré intégrale à valeur de $\R$ dans $\C^n$, et l'ensemble de tel signaux seront noté $L^2\big(\R,\C^n\big)$.
\\
Dans le cas $n=2$, on parle de signal \emph{bivarié}.
\end{definition}

\begin{proposition}\label{prop:quatern}
Les signaux bivariés se décrivent très simplement à l'aide des quaternions. En considérant $\{1, \bf{i},\bf{j},\bf{k}\}$ la base canonique des quaternions $\mathbb{H}$, on peut voir $\psi$ comme étant à valeur dans ${\C_{\bf{j}}}^n$ ($\C_{\bf{j}} :=\R\times \bf{j}\R$), de sorte que :
\[\forall \psi\in L^2(\R,\mathbb{H}),\ \exists a,\theta,\chi,\varphi \in\mathcal{C}(\R)\ |\quad \psi(t) = a(t)e^{\bf{i}\theta(t)}e^{-\bf{k}\chi(t)}e^{\bf{j}\varphi(t)}\]
\\
Sous cette forme, les paramètres $a$ et $\varphi$ s'interprètent respectivement comme l'amplitude et la phase instantanée du signal. Les deux paramètres restant contrôle l'ellipticité ($\chi$) et l'orientation ($\theta$) de l’ellipse de polarisation instantanée. C'est-à-dire l'ellipse que suit la signal à l'instant $t$.
\\
Dit autrement, à tout instant $t$, $\psi(t)$ est vu comme une point d'une ellipse dont la taille est caractériser par $a(t)$, l'ellipticité par $\chi(t)$ et l'orientation par $\theta(t)$. $\phi(t)$ permet lui de situer $\varphi(t)$ sur cette ellipse.
\\

\textit{Le problème de cette représentation est qu'elle se généralise mal aux signaux plus que $2-$variés et, à notre connaissant, il n'existe pas d'extensions des quaternions à de plus haute dimension. voir \cref{prop:gene_param_signal_v1,prop:gene_param_signal_v2}, \cref{eq:phase_tot,,eq:phase_dyn,eq:phase_geo}} 
\end{proposition}

Il est évident que cette représentation est présent bien plus de paramètre que nécessaire, puisse que deux devrait suffire. Pour autant, elle permet de mieux \textbf{je sais quoi mais c'est sur qu'il y'a une raison}.
\\
Si cette représentation se généralise mal parce qu'elle demanderait d'avoir une extension de $\mathbb{H}$, sont interprétations graphique, elle, se généralise très bien. Par exemple, en dimension 3, alors l'ellipse devient une ellipsoïde. L'amplitude reste de dimension 1 parce qu'elle ne fait que contrôler la taille de cet ellipsoïde, mais les autres paramètres eux doivent être de dimension 2. L'ellipsoïde à besoin de deux angles pour être orienté, possède deux degrés d'ellipticité et ces points sont déteminés par deux angles.
\\

\begin{proposition}\label{prop:gene_param_signal_v1}
Plus généralement, tout signal multivarié $\psi$ est (\textit{devrait être}) caractérisé par quatre paramètres (donc $1+(n-1)(\frac{n}{2}-2)$ scalaires) :
\begin{align*}
	a&\in\mathcal{C}(\R,\R^+)  &  \theta&\in\mathcal{C}(\R, [-\pi/2,\pi/2]^{\frac{n(n-1)}{2}})  &  \chi&\in\mathcal{C}(\R, [-\pi/4,\pi/4]^{n-1})  &  \varphi&\in\mathcal{C}(\R, [-\pi,\pi]^{n-1})
\end{align*}	
\end{proposition}

\`A bien y réfléchir, décrire un ellipsoïde dans l'espace, c'est exactement de que font les matrices symétriques définies positives. Donc on pourrait tout à fait remplacer les informations $(a,\theta,\chi)$ par une matrice symétrique positive de dimension $n$. Il ne resterait alors plus que $\varphi$ qui, de toute façon ne devrait pas trop être lié aux autres paramètres.

Enfin, surement que si parce que y'a un monde pour $\varphi=0_\R^n$ et c'est le reste des paramètres qui fait le travail. Mais clairement c'est pas intéressant comme description. L'idée serait plutôt décrire le signal $\psi$ en minimisant les variations de $(a,\theta,\chi)$.
Ca appelle clairement à chercher que dans l'espace de Siegel mais pas seulement, parce que c'est pas juste des chemins chez Siegel qui nous intéresse.

Ou alors c'est le jeu de jauge qui fait qu'on tue $\varphi$ ? auquel cas tout les jours Siegel.
\\

\textit{BTW, les quaternions c'est fait pour décrire les rotations et c'est (quasiment) ce qu'on fait avec, donc aller chercher dans un espace de matrices pour généraliser le principe c'est pas déconnant.}
\\
\textit{D'ailleurs, vu que c'est pas exactement ce qu'on fait avec, dans quelle mesure c'est pas le cas et est-ce qu'on exploite vraiment la structure des quaternions ?}
\\ 

\begin{proposition}\label{prop:gene_param_signal_v2}
Autre approche : un signal multivarié étant moralement un chemin de $\R^n$, son graphe est une variété (plongée) de dimenion 1. Sachant cela, si en chaque instant on veut définir l'ellipsoïde sur laquelle elle repose à un insant $t$, il est morale que cette ellipsoïde soit en fait une ellipse puisque c'est elle-même une variété de dimension 1.
\\
Partant de là, on aurait toujours $a$, $\chi$ et $\phi$ pour la décrire et seulement $\theta$ gagnerait en dimension pour pouvoir orienter l'ellipse dans les $n$ axes. $\psi$ serait alors la données de $3+\frac{n(n-1)}{2}$ paramètres :
\begin{align*}
	a&\in\mathcal{C}(\R,\R^+)  &  \theta&\in\mathcal{C}(\R, [-\pi,\pi]^{\frac{n(n-1)}{2}})  &  \chi&\in\mathcal{C}(\R, [-\pi/4,\pi/4])  &  \varphi&\in\mathcal{C}(\R, [-\pi,\pi])
\end{align*}
\end{proposition}

On aurait beaucoup moins de paramètre et c'est quand-même bien. En même temps ca parait plus contraignant comme modèle. Pour comparer les deux, il faudrait voir comment les deux se décomposant dans le cas d'un signal qui ne varierait sur une ellipsoïde fixe. \ie dans un cas où $\theta,\chi$ de la \cref{prop:gene_param_signal_v1} varie pas alors que ceux de la \cref{prop:gene_param_signal_v2} si.




\setcounter{figure}{0}
\setcounter{lstlisting}{0}

\section{Phase géométrique d'un signal}\label{sec:phasegeo}

La phase géométrique est invariante par action du groupe $\mathbb{U}(1)$, c'est à dire invariante par changement de la phase instantanée. Elle ne regarde donc que les paramètres $a, \theta$ et $\chi$. En outre, elle ne regarde que la matrice semi-définie positive (potentiellement sans même regarder l'amplitude $a$ a.k.a. la norme de la matrice).

Avec cette formulation il devient plus clair que $\phi$ peut être vu comme un fibré vectoriel (particulier parce que c'est bien un élément du produit $\R\times\C^n$ pas d'un espace qui s'en rapproche localement) et cette et que le groupe $\mathbb{U}(1)$ agit comme un jauge. En plus, Wikipédia dit que pour les fibrés vectoriels, il faut que l'action soit les changement de base (orthonormée ?), ce qui correspond bien à une changement de phase suivant les formules de la \cref{prop:gene_param_signal_v1}.

La question reste encore de savoir comment on détermine les $(a,\theta,\chi,\varphi)$ parce que encore une fois, y'a vraiment pas unicité de la décomposition et c'est loin d'être clair comment choisir la plus pertinente (et comment la calculer aussi !)



\section{Trucs à voir}

\subsection{Bilan des formules}

\begin{itemize}
	\item Les phases de $\psi$ entre les instants $t_0$ et $t$ :
	\begin{equation}\label{eq:phase_tot}
		\phaset(\psi, t_0, t) \equiv \arg(\psi(t_0), \psi(t))
	\end{equation}
	
	\begin{equation}\label{eq:phase_dyn}
		\phased(\psi,t_0,t) \equiv \Im m \int_{t_0}^t\big\langle \psi(s) \,|\, \dot{\psi}(s) \big\rangle ds
	\end{equation}
	
	\begin{equation}\label{eq:phase_geo}
		\phaseg(\psi, t_0, t) \equiv  \phaset(\psi, t_0, t) - \phased(\psi,t_0,t)
	\end{equation}
	
	\item (conservative) Équation Schrödinger et de Liouville-von Neumann ($h(R)$ : Hamiltonien des paramètres $R$, $W$ : opérateurs statistique ) \cite[p.6]{bohm_geometric_2003} :
	\begin{equation}\label{eq:schrodinger}
		i\frac{d \psi(t)}{dt} = h(R)\psi(t)
	\end{equation}
	\begin{equation}\label{eq:liouville-neumann}
		i\frac{d W(t)}{dt} = \big[h(R),W(t)\big] \qquad\qquad [\cdot\,,\cdot]=\text{ commutateur ?}
	\end{equation}
	
	\item Moment angulaire (viteuf) $\forall z\in\C$ :
	\begin{equation}\label{eq:mom_angu}
		M(t) = \Re e \big(iz\overline{z}'\big) = -\Im m z\overline{z}'  \qquad\qquad \text{thoughts ?}
	\end{equation}
	
\end{itemize}


\subsection{Général}

\begin{itemize}
	\item D'où sort l'interprétation géométrique + son lien avec quaternions (prop. \ref{prop:quatern})
	\item Lien avec l'eq de Schrödinger (l'intérêt de $H$ et $K$ + d'où ils sortent)
	\item J'ai rien compris au ``problème'' des formalisations vecteur et/vs complexe
	\item Comment comprendre $\big\langle \psi\, |\, \dot{\psi} \big\rangle$ ?
	
	\item La ``Berry connection'' c'est une vraie connexion ? elle est où la covariance alors ?
\end{itemize}

\[\underline{\overline{\qquad\qquad\qquad\qquad\qquad\qquad\qquad\qquad\qquad\qquad\qquad\qquad\qquad\qquad\qquad\qquad\qquad\qquad}}\]{\color{white}relbinrei}

\begin{itemize}
	\item ``horizontal lift'' : pourquoi horizontal ? en quel sens ?
	\item Fréquence de Rubi
	\item Matrice/base de Pauli et généralisation, groupe $SU(n)$ (un peu de quantique ?)
	\item Produit hermitien : intuition géométrique
	\item Monopole de Dirac + lien avec la phase géo (un peu d'électro-magnétisme ?)
	\item Invariant de Bargmann + série de Dyson
\end{itemize}



\subsection{Point de vue des variétés}\label{subsec:phaseG_variete}

\begin{itemize}
	\item Ecriture en terme de fibré (principale ? vectoriel ?)
	
	\item Choix de l'action de groupe pour la jauge : $U(1)$ \apriori
\end{itemize}

\[\underline{\overline{\qquad\qquad\qquad\qquad\qquad\qquad\qquad\qquad\qquad\qquad\qquad\qquad\qquad\qquad\qquad\qquad\qquad\qquad}}\]{\color{white}relbinrei}

\begin{itemize}
	\item Lien avec Siegel : assez clair avec la visualisation des ellipses, beaucoup moins avec les Hilberts même si $| \psi\rangle \langle \psi |\in$ Siegel \apriori
	
	\item ``\textit{Symplectique}'' (meaning + intérêt ?)
	
	\item 
\end{itemize}




\newpage

\listoffigures
\vfill
\lstlistoflistings
\vfill
%\listtheorems

\newpage

\bibliography{ref.bib}{}
\bibliographystyle{siam}
\end{document}