
%%%%    IMPORTS 2 BASE    %%%%


\documentclass[hidelinks, french, oneside]{article}
\usepackage[utf8]{inputenc}
\usepackage[T1]{fontenc}

% pour le mise en page
\usepackage[a4paper, total={6.5in, 10in}]{geometry}
\usepackage{fontsize} \changefontsize[12]{10}		
\usepackage{xcolor}

% mathsymbole
\usepackage{amsmath, amssymb,stmaryrd}
\usepackage{wasysym}[mathcal] % a quoi sert le [mathcal] ?
\usepackage[nointlimits]{esint} % pour \fint, rien d'autre
\usepackage{nicefrac, units} % fraction pour text mode / unité
%\usepackage{cancel} % pour les simplifications de calcul (trop nice)

% TBD
\usepackage{mathcomp}
\usepackage{mathrsfs} % pour \mathscr a priori

% pour les belles fonts
\usepackage{amsfonts}
\DeclareMathAlphabet{\mathpzc}{OT1}{pzc}{m}{it}
%\usepackage{euscript}[mathcal]
%\usepackage{rsfso}
\usepackage{bbm}	 % mathbb étendu

% pour les hyprlien / cross-ref
\usepackage{hyperref, cleveref}
%\namecref{sec:label}

%pour les figures
\usepackage{graphicx, caption} \usepackage{wrapfig, floatrow}
%\usepackage{subcaption} % surement useless, parsk floatrow-like 

% pour les beaux tableaux
\usepackage{multirow}
\renewcommand{\arraystretch}{1.2}

% pour des matrices infernales
\usepackage{easybmat}

% pour les citations (aucune idées de comment ca marche)
\usepackage{csquotes}



%%%%    RACCOURCIS    %%%%


% bb / cal / frak
\newcommand{\N}{\mathbb{N}}
\newcommand{\Z}{\mathbb{Z}}             % Note pour homo pour les updates
\newcommand{\Q}{\mathbb{Q}}
\newcommand{\R}{\mathbb{R}}
\newcommand{\C}{\mathbb{C}}
\newcommand{\K}{\mathbb{K}}
\renewcommand{\k}{\Bbbk}
\newcommand{\U}{\mathbb{U}}
\renewcommand{\u}{\text{U}}
\newcommand{\A}{\mathbb{A}}
\newcommand{\T}{\mathscr{T}}
\newcommand{\I}{\mathbb{I}}
\newcommand{\one}{\mathbbm{1}}
%\renewcommand{\S}{\mathfrak{S}}
\renewcommand{\S}{\mathbb{S}}

% arrows
\newcommand{\lr}{\longrightarrow}
\newcommand{\Lr}{\Longrightarrow}
%\renewcommand{\ll}{\longleftarrow}
\newcommand{\Ll}{\Longleftarrow}
\newcommand{\llr}{\longleftrightarrowr}
\newcommand{\Llr}{\Longleftrightarrow}

% espaces
\newcommand{\matk}{\mathpzc{M}_n(\mathbb{K})}
\newcommand{\matr}{\mathpzc{M}_n(\mathbb{R})}

% fonctions
\newcommand{\Arccos}{\text{\normalfont Arccos}} 
\newcommand{\Arcsin}{\text{\normalfont Arcsin}} 
\newcommand{\Arctan}{\text{\normalfont Arctan}} 
\newcommand{\Argch}{\text{\normalfont Argch}}       
\newcommand{\Argsh}{\text{\normalfont Argsh}}

\newcommand{\congu}[1]{\overline{#1}}
\newcommand{\argmin}[1]{\underset{#1}{\text{\normalfont argmin}}}
\newcommand{\argmax}[1]{\underset{#1}{\text{\normalfont argmax}}}

\newcommand{\pgcd}{\text{\normalfont pgcd}}
\newcommand{\PGCD}{\text{\normalfont PGCD}}
\newcommand{\ppmc}{\text{\normalfont ppcm}}
\newcommand{\sign}{\text{\normalfont sign}}

\newcommand{\sgn}{\text{\normalfont sgn}}
%\newcommand{\deg}{\text{deg}}
\newcommand{\ord}{\text{\normalfont ord}}
\newcommand{\rot}{\text{\normalfont rot}}

%\renewcommand{\det}{\text{det}}
\newcommand{\tr}{\text{\normalfont tr}}
\newcommand{\rg}{\text{\normalfont rg}}
\newcommand{\Co}{\text{\normalfont com}}
\newcommand{\codim}{\text{\normalfont codim}}

\newcommand{\Fou}[1]{\mathcal{F}\left[#1\right]}
\newcommand{\fou}[1]{\hat{#1}}

%espaces
%\renewcommand{\Vec}{\text{Vec}}
\newcommand{\im}{\text{\normalfont Im}}
\newcommand{\Ker}{\text{\normalfont Ker}}
\newcommand{\Ann}{\text{\normalfont Ann}}
\newcommand{\Sp}{\text{\normalfont Sp}} 
\newcommand{\GL}{\text{\normalfont GL}}
\newcommand{\SL}{\text{\normalfont SL}}
\newcommand{\SO}{\text{\normalfont SO}}
\newcommand{\SU}{\text{\normalfont SU}}
%\renewcommand{\div}{\text{div}}

\newcommand{\Aff}{\text{\normalfont Aff}}
\newcommand{\HT}{\text{\normalfont HT}}
\newcommand{\GA}{\text{\normalfont GA}}

% spé proba
\newcommand{\esp}[2][]{\mathbb{E}_{#1}\left[\, #2\, \right]}
\newcommand{\var}[2][]{\mathbb{V}_{#1}\left[\, #2\, \right]}

% plus jolie
\renewcommand{\O}{\varnothing}
\renewcommand{\epsilon}{\varepsilon}
\renewcommand{\subsetneq}{\varsubsetneq}
\renewcommand{\leq}{\leqslant}
\renewcommand{\geq}{\geqslant}
\renewcommand{\limsup}{\varlimsup}
\renewcommand{\liminf}{\varliminf}

% autre
\newcommand{\defeq}{:=}
\renewcommand{\bf}[1]{\boldsymbol{#1}}
\renewcommand{\AC}{\sim}
\newcommand{\para}{\sslash}
\newcommand{\barint}[2][]{\text{\sout{\ensuremath{\displaystyle \int_{#2}}}}^{#1}}

% latin
\newcommand{\etal}{\textit{et al.}}
\newcommand{\etc}{\textit{etc.}}
\newcommand{\apriori}{\textit{a priori}}
\newcommand{\afortiori}{\textit{a fortiori}}
\newcommand{\acontrario}{\textit{a contrario}}
\newcommand{\infine}{\textit{in fine}}
\newcommand{\ie}{\textit{i.e.}}
\newcommand{\eg}{\textit{e.g.}}

% tmp
\newcommand{\phaset}{\Phi_{\text{\normalfont tot}}}
\newcommand{\phased}{\Phi_{\text{\normalfont dyn}}}
\newcommand{\phaseg}{\Phi_{\text{\normalfont geo}}}
\renewcommand{\i}{i}
\newcommand{\SA}[1]{\mathcal{A}\left[#1\right]}
\newcommand{\sa}[1]{\bf{#1}_+}
\newcommand{\hilb}[1]{\mathcal{H}\left[#1\right]}
\newcommand{\vpC}{\text{\normalfont vp} \frac{1}{x}}


%%%%    ENONCES (PROP, DEF, RQ)    %%%%


%\usepackage{amsthm}
\usepackage[hyperref, noconfig]{ntheorem}
\usepackage[ntheorem, framemethod=tikz]{mdframed}


% les styles
\definecolor{colorbar}{rgb}{1.0, 0.55, 0.0}
\mdfdefinestyle{barred}{leftmargin=0.1cm, skipabove=15pt, skipbelow=0pt, 	% tailles des marges
	linewidth=0.1cm, linecolor=colorbar, 	% box' style
	topline=false, bottomline=false, rightline=false}	% coté de la box

\mdfdefinestyle{demo frame}{leftmargin=0.15cm, rightmargin=0.7cm, 	
	linewidth=0.4pt, linecolor=black,
	topline=false, bottomline=true, rightline=false}


% énoncés classiques
\makeatletter
	\newtheoremstyle{enonce}{##1 ##2}{##1 ##2 (##3)}
\makeatother
\theoremheaderfont{\scshape}
\theoremseparator{\quad ---\quad}
\theoremnumbering{arabic}
\theorembodyfont{\normalfont}

\newmdtheoremenv[style=barred]{definition}{Définition}%[section]
\newmdtheoremenv[style=barred]{proposition}{Proposition}
\newmdtheoremenv[style=barred]{propriete}{Propriété}
\newmdtheoremenv[style=barred]{propricarac}[propriete]{Propriété Caractéristique}
\newmdtheoremenv[style=barred]{lemme}{Lemme}
\newmdtheoremenv[style=barred]{theoreme}{Théorème}
\newmdtheoremenv[style=barred]{theodef}[theoreme]{Théorème et Définition}
\newmdtheoremenv[style=barred]{corollaire}{\indent Corollaire}[theoreme]

% énoncé type
\makeatletter
	\newtheoremstyle{special}{##1 ##2}{{\scshape ##3}\quad ---\quad}
\makeatother
\theoremstyle{special}
\theoremheaderfont{\scshape}
\theoremseparator{\quad ---\quad}
\theoremnumbering{arabic}
\theorembodyfont{\normalfont}

\newmdtheoremenv[style=barred]{enonce}{}

% énoncé numérotation-less
\theoremstyle{nonumberplain}
\theoremheaderfont{\normalfont\scshape}
\theoremseparator{\quad ---\quad}
\theoremnumbering{arabic}
\theorembodyfont{\normalfont\itshape}

\newtheorem{remarque}{\indent Remarque}
\newtheorem{rappel}{\indent Rappel}
\newtheorem{exemple}{\indent Exemple}

% exos
\theoremstyle{break}
\theoremheaderfont{\bfseries\scshape\color{blue}}
\theoremseparator{ : }
\theoremnumbering{arabic}
\theorembodyfont{\normalfont}
\newtheorem{exercice}{Exercice}

% demo
\theoremstyle{empty}
\theoremheaderfont{\itshape}
\theoremseparator{\quad ---\quad}
\theoremnumbering{arabic}
\theorembodyfont{\itshape}

\newmdtheoremenv[style=demo frame]{demonstration}{Démonstration}

\newenvironment{demo}{%
	\vspace{0.5cm} \hspace{-0.55cm}\textbf{\textit{Démonstration}} \qquad \rule{5cm}{0.4pt}
	\begin{demonstration}%
	}{\newline \rule{15.3cm}{0.4pt} \end{demonstration}}


	% pour cref aux enoncés
	
\crefname{definition}{définition}{Définition}
\crefname{proposition}{proposition}{Proposition}
\crefname{propriete}{propriété}{Propriété}
\crefname{lemme}{lemme}{Lemme}
\crefname{theoreme}{théorème}{Théorème}
\crefname{corollaire}{corollaire}{Corollaire} 

\crefname{remarque}{remarque}{Remarque}
\crefname{rappel}{rappel}{Rappel}
\crefname{exemple}{exemple}{Exemple}

\crefname{exercice}{exercice}{Exercice}



%%%%    PROG N PLOT    %%%%


% Prog

%\usepackage{fancyvrb} % pour le lore, mais useless AF
\usepackage{listings} 
%\usepackage{pythontex}	% surement une dingz, à approfondire

% preset
\definecolor{codeblack}{gray}{0.2}
\definecolor{codegreen}{rgb}{0,0.6,0}
\definecolor{codegray}{gray}{0.5}
\definecolor{codepurple}{rgb}{0.58,0,0.82}
\definecolor{backcolour}{rgb}{0.98,0.98,0.95}

\lstdefinestyle{mystyle}{
backgroundcolor=\color{backcolour},   
commentstyle=\color{codegray},
keywordstyle=\color{orange},
numberstyle=\tiny\color{codegray},
stringstyle=\color{codegreen},
basicstyle=\color{codeblack}\ttfamily\footnotesize,
breakatwhitespace=false,         
breaklines=true,                 
captionpos=b,
abovecaptionskip=12,
belowcaptionskip=18,
keepspaces=true,                 
numbers=left,                    
numbersep=5pt,                  
showspaces=false,                
showstringspaces=false,
showtabs=false,                  
tabsize=2,
frame=single,
rulecolor=\color{lightgray}}
\lstset{style=mystyle}


% Plot

% pour dessins et les diagrams
\usepackage{tikz, tikz-cd, tkz-euclide}
\usepackage{scalerel, pict2e}
\usetikzlibrary{calc, patterns, shapes.arrows, arrows.meta, shadows, external} 

% pour les maxis plots
\usepackage{pgfplots}
\pgfplotsset{compat=newest, scaled y ticks=false}
\usepgfplotslibrary{groupplots, dateplot, statistics, fillbetween}

\tikzstyle{every node}=[font=\scriptsize]
\pgfplotsset{standard/.style={width=8cm,
	height=3cm,
	compat=1.18,
	trig format=rad,
	enlargelimits,
	axis x line=middle,
	axis y line=middle,
	enlarge x limits=0.15,
	enlarge y limits=0.15,
	every axis x label/.style={footnotesize, at={(current axis.right of origin)},anchor=north west},
	every axis y label/.style={footnotesize, at={(current axis.above origin)},anchor=south east},
	scale only axis=true}}



%%%%    CAPTIONS 2 FIGURES    %%%%


% set up des captions figures (extrêmement BG)

% pos et séparateur générale
\captionsetup{justification=centering}
\DeclareCaptionLabelSeparator{custom}{\, ---\, }
%\DeclareCaptionFormat{custom}{#1#2#3}

% spécial figure
\DeclareCaptionLabelFormat{customfig}{fig. #2}
\renewcommand{\thefigure}{\itshape\arabic{part}.\arabic{figure}}
\captionsetup[figure]{font=it, labelformat=customfig, labelsep=custom}

% spécial tableau
\DeclareCaptionLabelFormat{customtab}{tab. #2}
%\renewcommand{\thetable}{\itshape\arabic{part}.\arabic{figure}}
\captionsetup[table]{font=it, labelformat=customtab, labelsep=custom}

% spécial code (listings)
\DeclareCaptionLabelFormat{customcode}{\textit{code \arabic{section}.#2}}
%\renewcommand{\thelstlisting}{\arabic{part}.\arabic{lstlisting}}
\captionsetup[lstlisting]{labelformat=customcode, labelsep=custom}

% pour cref
\crefname{figure}{\itshape fig.}{Figure}
\crefname{table}{\itshape tab.}{Table}
	
	
	%%%%    TABLES/REF  &  HAUT/BAS DE PAGES    %%%%
	
	
	% nom des tables et références
	
	\renewcommand{\contentsname}{\begin{center}\textsc{Tables des Matrières}\end{center}}
	\renewcommand{\listfigurename}{\begin{center}\textsc{Table des Figures}\end{center}}
	\renewcommand{\lstlistlistingname}{\begin{center}\textsc{Table des Codes}\end{center}}
	\renewcommand{\refname}{\begin{center}\textsc{Références}\end{center}}
	
	% bas/haut de page (à optimiser à l'occas')
	
	\usepackage{fancyhdr}
	\pagestyle{fancy}                       
	\fancyhf{}
	\renewcommand{\headrulewidth}{0pt}
	\cfoot{\thepage}
	
	%version2réjane  :
	%\usepackage{eso-pic}
	%\renewcommand\headrulewidth{1pt}
	%\fancyhead[L]{\includegraphics[width=3cm]{logo_lr.png}}
	%\fancyhead[R]{\includegraphics[width=1cm]{logo_sdis17.png}}
	%\setlength{\headsep}{45pt}
	
	
	
	%%%%    SECTIONS ET TOC    %%%%
	
	
	% génération des sections
	
\usepackage[loadonly, clearempty, newparttoc, toctitles]{titlesec}

% liste des sections
\titleclass{\part}[0]{top} % 0 = niveau de section
\titleclass{\section}{straight}[\part]
\titleclass{\subsection}{straight}[\section]
\titleclass{\subsubsection}{straight}[\subsection]

\makeatletter % pour que les sections suivent les parts
	\@addtoreset{section}{part} % (jsp pk ca marche pas sinon :/)
\makeatother

% génère les numérotations avec format
%\newcounter{part}
\renewcommand{\thepart}{\Roman{part}}
%\newcounter{section}
\renewcommand{\thesection}{\Roman{section}}
%\newcounter{subsection}
\renewcommand{\thesubsection}{\arabic{section}.\arabic{subsection}}
%\newcounter{subsubsection}
\renewcommand{\thesubsubsection}{\arabic{section}.\arabic{subsection}.\arabic{subsubsection}}

% formatage
\titleformat{\part}[display]{\bfseries\scshape\Large}{\centering \rule{3.5cm}{0.4pt}\qquad Partie\quad \thepart \qquad \rule{3.5cm}{0.4pt}}{15pt}{\centering}[\vspace{0.2cm}\rule{8cm}{0.4pt}]
\titlespacing{\part}{0pt}{50pt}{70pt}
\newcommand{\partbreak}{\clearpage}

\titleformat{\section}{\bfseries\Large}{\thesection\quad ---}{15pt}{}
\titlespacing{\section}{10pt}{20pt}{15pt}
%\newcommand{\sectionbreak}{\clearpage}

\titleformat{\subsection}{\bfseries\large}{\thesubsection}{15pt}{}
\titlespacing{\subsection}{20pt}{15pt}{15pt}

\titleformat{\subsubsection}{\bfseries}{\thesubsubsection}{15pt}{}
\titlespacing{\subsubsection}{30pt}{15pt}{15pt}

% pour cref aux enoncés
\crefname{part}{partie}{Partie}
\crefname{section}{section}{Section}
\crefname{subsection}{sous-section}{Sous-section}
\crefname{subsubsection}{sous-section}{Sous-section}


	% le TOC en lé-gende

\usepackage{titletoc}
%\contentsmargin{2em}

\titlecontents{part}[2em]{\bfseries\large\scshape 
	\hspace*{-1.5em}\rule{\textwidth}{0.5pt}\vspace{0.1cm}\\*
	 }{Partie\quad \contentslabel{-0.05em}\quad\ ---\quad}{\contentslabel{-1.5em} ---\quad }{\hfill\contentspage}[\hspace*{-0.5em}
\rule{\textwidth}{0.5pt}]

\titlecontents{section}[1.5em]{\addvspace{1em}\bfseries}{\contentslabel{1.25em} ---\quad}{\hspace*{-1.75em}}{\titlerule*[0.75pc]{.}\contentspage}[\addvspace{0.25em}]

\titlecontents{subsection}[3.8em]{\addvspace{0.15em}\normalfont}{\contentslabel{2em}}{\hspace*{-2em}}{\titlerule*[0.75pc]{.}\contentspage}

\titlecontents{subsubsection}[6.8em]{\normalfont}{\contentslabel{2.75em}}{\hspace*{-2.5em}}{\titlerule*[0.75pc]{.}\contentspage}


% set up d'un environnement pour les annexes

\newenvironment{annexe}{%
	\newpage
	
	% changement title sec
	\titleformat{\section}[display]{\bfseries\scshape\Large}{\centering}{15pt}{\centering}
	\titlespacing{\section}{0pt}{30pt}{40pt}
	
	\titleformat{\subsection}{\bfseries\large}{Annexe \thesubsection\quad ---\quad}{0pt}{}
	
	\titleformat{\subsubsection}{\bfseries}{\thesubsubsection}{15pt}{}
	
	% changement title toc
	\titlecontents{section}[0.25em]{\addvspace{0.5em}\bfseries}{}{\hspace*{-1.5em}}{\titlerule*[0.75pc]{.}\contentspage}
	
	\titlecontents{subsection}[1.5em]{\normalfont Annexe\hspace*{2.5em}}{\contentslabel{2em}}{\hspace*{-2em}}{\titlerule*[0.75pc]{.}\contentspage}
	
	\titlecontents{subsubsection}[6em]{\normalfont}{\contentslabel{2.75em}}{\hspace*{-2em}}{\titlerule*[0.75pc]{.}\contentspage}
	
	% changment de la numéritations
	\renewcommand{\thesubsection}{\Alph{subsection}}
	\renewcommand{\thesubsubsection}{\Alph{subsection}.\arabic{subsubsection}.}
	% mise à zéro des compters
	%\setcounter{section}{0}
}{}


%POUR NTHEOREM (A FAIRE ET A FIX)
% vertical bar pour les énonces : https://tex.stackexchange.com/questions/499011/ntheorem-framing-environment-with-vertical-lines-on-both-the-left-and-right-sid